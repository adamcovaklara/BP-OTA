\noindent Cílem práce bylo navrhnout pásmovou propust 4. řádu s Cauerovou aproximací. Po seznámení s principy OTA \ref{s:OTA} a náhradou prvků v obvodech s nimi \ref{s:NAH} byla provedena simulace. V sekci \ref{s:DP2} bylo v MultiSimu realizováno zapojení filtru typu dolní propust 2. řádu s poklesem 40 dB/dek, pásmová propust 1. řádu s poklesem 20 dB/dek. Poté byl kaskádním zapojením obdržen filtr typu dolní propust 4. řádu s poklesem 80 dB/dek, pásmová propust 2. řádu s poklesem 40 dB/dek. Dalším kaskádním blokem byla obdržena požadovaná pásmová propust 4. řádu s poklesem 80~dB/dek. Výsledky simulací prokazují poměrně dobré vlastnosti navržené struktury.\\
\\
V sekci \ref{s:MAPLE} byla knihovnou Syntfil provedena matematická syntéza filtru a zapojení bylo převedeno na LC příčkovou strukturu. Mezní kmitočet a parametry propustného a zádržného pásma byly zvoleny v řádech stovek kHz. Zesílení je možné řídit, stejně tak mezní kmitočet. Mezikrokem v návrhu byl převod pásmové propusti na normovanou dolní propust. Pro LC strukturu byly obdrženy odnormované hodnoty prvků, které byly dosazeny do zapojení s OTA a byla provedena simulace s vypočtenými prvky (sekce \ref{s:SIMAPLE}). Výsledné zapojení ze sekce \ref{s:DP2} obsahovalo 8 kapacitorů. ARC syntézou v sekci \ref{s:ARC}, která spočívá v náhradě indukčností gyrátory, bylo z LC příčkového filtru ze sekce \ref{s:MAPLE} získáno 10 kapacitorů. Kvůli použití pouze rezistorů a kapacitorů jsou filtry realizované kaskádní syntézou výhodnější, protože induktory se musí složitě vyrábět na danou hodnotu. Výsledné schéma celkem obsahuje 12 pasivních a 13 aktivních komponent. V simulaci vycházející z ARC syntézy byl vstupní odpor řazen paralelně ke zdroji. Je nutné dbát na to, že toto zapojení obsahuje neuzemněné kapacity a nebude vhodné pro krátké vlny (frekvence v řádech MHz, což odpovídá vlnovým délkám 10 -- 100~m). Pro tyto vysoké frekvence také OTA nemohou být použity kvůli limitovanému \textit{GBP}. U neuzemněných kapacit je také nutné dbát na to, že vstupní externí proud může způsobit akumulaci náboje na kapacitorech a eventuálně i saturaci OTA. Větší počet OTA také kvůli zpětným vazbám může mít vliv na stabilitu celého obvodu, čímž se sníží pásmo pro vstupní externí proud -- filtr pak může být stabilní jen v malém kmitočtovém pásmu.\\
\\
Dalším krokem byl návrh v Altiu s dvěma různými konektory (sekce \ref{s:PRAK}), praktická realizace a odzkoušení navrhnutého obvodu.\\
Pro lepší návrh by bylo vhodné třeba analyzovat výslednou strukturu popsanou přenosy gyrátorů (sekce \ref{s:MAPLE}) a získat z ní zapojení s OTA, nebo také analyzovat výsledný obvod knihovnou Pracan (netlist struktury lze vyexportovat z Multisimu nebo Altia).