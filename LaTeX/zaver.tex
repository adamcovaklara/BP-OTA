\noindent Cílem práce bylo navrhnout pásmovou propust 4. řádu s Cauerovou aproximací. Po seznámení s principy OTA \ref{s:OTA} a náhradou prvků v obvodech s nimi \ref{s:NAH} byla provedena simulace. V sekci \ref{s:DP2} bylo v MultiSimu realizováno zapojení filtru typu dolní propust 2. řádu s poklesem 40~dB/dek, pásmová propust 1. řádu s poklesem 20~dB/\mbox{dek}. Poté byl kaskádním zapojením obdržen filtr typu dolní propust 4. řádu s poklesem 80~dB/dek, pásmová propust 2. řádu s poklesem 40~dB/dek. Dalším kaskádním blokem byla obdržena požadovaná pásmová propust 4. řádu s poklesem 80~dB/dek. Výsledky simulací prokazují poměrně dobré vlastnosti navržené struktury.\\
\\
V sekci \ref{s:MAPLE} byla knihovnou Syntfil provedena matematická syntéza filtru a zapojení bylo převedeno na LC příčkovou strukturu. Mezní kmitočet a parametry propustného a zádržného pásma byly zvoleny $f\_s = 60$~kHz, $f\_p = 150$~kHz, fp = 190~kHz, fs = 280~kHz, kde $f\_s, fs$ označuje spodní a horní hranici nepropustného pásma a $f\_p, fp$ spodní a horní hranici propustného pásma. Útlum v propustném pásmu byl zvolen 1~dB a v zádržném 80~dB. Mezikrokem v návrhu byl převod pásmové propusti na normovanou dolní propust. Pro LC strukturu byly obdrženy hodnoty prvků, které byly odnormovány v sekci \ref{s:ARC}. Byla provedena ARC syntéza s využitím gyrátorů, po níž výsledný obvod obsahoval pouze OTA a kapacitory. Výsledné zapojení ze sekce \ref{s:DP2} obsahovalo 8 kapacitorů. Výsledným zapojením, vycházejícím z LC příčkového filtru, bylo získáno 10 kapacitorů. Výsledné schéma celkem obsahuje 15 aktivních součástek. Byla provedena THD analýza a porovnání šumu pro různé frekvence. Z této analýzy se potvrdily dobré propustné vlastnosti filtru na kmitočtu 100~kHz. Tento kmitočet byl pro účely simulace přeladěn klidovým stejnosměrným pracovním proudem z původních 169~kHz. \\
\\
Je nutné dbát na to, že toto zapojení obsahuje neuzemněné kapacity a nebude vhodné pro krátké vlny (frekvence v řádech MHz, což odpovídá vlnovým délkám 10--100~m). Pro tyto vysoké frekvence také OTA nemohou být použity kvůli limitovanému \textit{GBP}. U neuzemněných kapacit je také nutné dbát na to, že klidový stejnosměrný pracovní proud může způsobit akumulaci náboje na kapacitorech a eventuálně i saturaci OTA. Větší počet OTA také kvůli zpětným vazbám může mít vliv na stabilitu celého obvodu, čímž se sníží pásmo pro klidový stejnosměrný pracovní proud - filtr pak může být stabilní jen v malém kmitočtovém pásmu.\\
\\
Dalším krokem byl návrh v Altiu (sekce \ref{s:PRAK}). Obvod bude prakticky realizován a odzkoušen. Pro lepší návrh by bylo vhodné analyzovat výslednou strukturu popsanou přenosy gyrátorů (sekce \ref{s:MAPLE}) a získat z ní zapojení s~OTA, což by minimalizovalo počet OTA ve struktuře. Také lze analyzovat výsledný obvod knihovnou Pracan (netlist struktury lze vyexportovat z Multisimu nebo Altia).