\noindent Cílem práce bylo navrhnout pásmovou propust. Po~seznámení s principy OTA \ref{s:OTA} a náhradou prvků v obvodech s nimi \ref{s:NAH} byla provedena simulace. V sekci \ref{s:DP2} bylo v MultiSimu realizováno zapojení filtru typu dolní propust 2. řádu s poklesem 40\,dB/dek, pásmová propust 1. řádu s poklesem 20\,dB/\mbox{dek}. Poté byl kaskádním zapojením obdržen filtr typu dolní propust 4. řádu s poklesem 80\,dB/dek, pásmová propust 2. řádu s poklesem 40\,dB/dek. Dalším kaskádním blokem byla obdržena požadovaná pásmová propust 4. řádu s poklesem 80\,dB/dek. Výsledky simulací prokazují poměrně dobré vlastnosti navržené struktury.\\
\\
V sekci \ref{s:MAPLE} byla knihovnou Syntfil provedena matematická syntéza filtru a zapojení bylo převedeno na LC příčkovou strukturu. Byl proveden výpočet pro filtr 4. řádu s Butterworthovou aproximací. Mezní kmitočet a parametry propustného a zádržného pásma byly zvoleny $f_{-s} = 100$\,kHz, $f_{-p} = 140$\,kHz, 
$f_p$ = 160\,kHz, $f_s$ = 200\,kHz, kde $f_{-s}, f_s$ označuje spodní a horní hranici nepropustného pásma a $f_{-p}, f_p$ spodní a horní hranici propustného pásma. Útlum v propustném pásmu byl zvolen 1\,dB a v zádržném 40\,dB. Mezikrokem v návrhu byl převod pásmové propusti na normovanou dolní propust. Pro LC strukturu byly obdrženy hodnoty prvků, které byly odnormovány v sekci \ref{s:ARC123}. Byla provedena ARC syntéza s využitím gyrátorů, po níž výsledný obvod obsahoval pouze OTA a kapacitory. Výsledným zapojením vycházejícím z LC příčkového filtru bylo získáno 8 kapacitorů a 12 OTA (6 bloků LM13700). Byla provedena THD analýza a porovnání šumu pro různé frekvence. Z této analýzy se potvrdily dobré propustné vlastnosti filtru na kmitočtu 100\,kHz. Tento kmitočet byl pro účely simulace přeladěn klidovým stejnosměrným pracovním proudem z původních 150\,kHz. \\
\\
Knihovnou Syntfil byl spočítán i filtr 8. řádu s Cauerovou aproximací. Mezní kmitočet a parametry propustného a zádržného pásma byly zvoleny $f_{-s} = 60$\,kHz, $f_{-p} = 150$\,kHz, 
$f_p$ = 190\,kHz, $f_s$ = 280\,kHz. Útlum v propustném pásmu byl zvolen 1\,dB a v zádržném 80\,dB. Výsledným zapojením vycházejícím z LC příčkového filtru bylo získáno 10 kapacitorů a 15 OTA. Byla provedena THD analýza a porovnání šumu pro různé frekvence.\\
\\
Je nutné dbát na to, že toto zapojení obsahuje plovoucí kapacity a nebude vhodné pro krátké vlny (frekvence v řádech MHz, což odpovídá vlnovým délkám 10--100\,m). Pro tyto vysoké frekvence také OTA nemohou být použity kvůli limitovanému GBP. U plovoucích kapacit je také nutné dbát na to, že klidový stejnosměrný pracovní proud může způsobit akumulaci náboje na kapacitorech a eventuálně i saturaci OTA. Větší počet OTA také kvůli zpětným vazbám může mít vliv na stabilitu celého obvodu, čímž se sníží pásmo pro klidový stejnosměrný pracovní proud --- filtr pak může být stabilní jen v malém kmitočtovém pásmu.\\
\\
Dalším krokem byl návrh DPS (sekce \ref{s:PRAK}) pro bikvady a PP 2. řádu. Obvod bude prakticky realizován a odzkoušen. Pro lepší návrh by bylo vhodné analyzovat výslednou strukturu popsanou přenosy gyrátorů (sekce \ref{s:MAPLE}) a získat z ní zapojení s~OTA, což by minimalizovalo počet OTA ve~struktuře.