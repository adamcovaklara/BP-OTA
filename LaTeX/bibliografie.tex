\newpage
\begin{thebibliography}{999}
\bibitem{1}
SHUENN-YUH, Lee a Cheng CHIH-JEN. \textit{Systematic Design and Modeling of a OTA-C Filter for Portable ECG Detection}. IEEE Transactions on Biomedical Circuits and Systems [online]. 2009, Únor 2009 (Vol. 3, no. 1), 11 [cit. 2019-11-05]. Dostupné z: \url{https://www.researchgate.net/publication/224367186_Systematic_Design_and_Modeling_of_a_OTA-C_Filter_for_Portable_ECG_Detection}
\bibitem{2}
\textit{Dlouhé vlny}. Wikipedia: the free encyclopedia [online]. San Francisco (CA): Wikimedia Foundation, 2001- [cit. 2019-11-04]. Dostupné z: \url{https://cs.wikipedia.org/wiki/Dlouh%C3%A9_vlny}
\bibitem{3}
\textit{Střední vlny}. Wikipedia: the free encyclopedia [online]. San Francisco (CA): Wikimedia Foundation, 2001- [cit. 2019-11-04]. Dostupné z: \url{https://cs.wikipedia.org/wiki/St%C5%99edn%C3%AD_vlny}
\bibitem{4}
\textit{ZigBee}. Wikipedia: the free encyclopedia [online]. San Francisco (CA): Wikimedia Foundation, 2001- [cit. 2019-11-04]. Dostupné z: \url{https://cs.wikipedia.org/wiki/ZigBee}
\bibitem{5}
HÁJEK, K., SEDLÁČEK, J. \textit{Kmitočtové  filtry}. Praha, BEN 2002, 536s. ISBN 80-7300-023-7.
\bibitem{6}
SUCHÁNEK, Tomáš. \textit{Kmitočtový filtr} [online]. Brno, 2009 [cit. 2019-11-05]. Dostupné z: \url{https://www.vutbr.cz/www_base/zav_prace_soubor_verejne.php?file_id=17738}. Bakalářská práce. VUT v Brně. Vedoucí práce Ladislav Káňa.
\bibitem{7}
KAŠPER, Ladislav. \textit{Návrh kmitočtového filtru} [online]. Ostrava, 2012 [cit. 2019-04-28]. Dostupné z: \url{https://dspace.vsb.cz/bitstream/handle/10084/92901/KAS279_FEI_N2647_2601T013_2012.pdf?sequence=1&isAllowed=y}. Diplomová práce. VŠB-TU Ostrava, FEI. Strana 18.
\bibitem{8}
VEDRAL, Josef a Jakub SVATOŠ. \textit{Zpracování a digitalizace analogových signálů v měřící technice}. Praha: Česká technika - nakladatelství ČVUT, 2018. ISBN 978-80-01-06424-5. Strana 136, obrázek 5.3.9, 5.3.10.
\bibitem{9}
HOSPODKA, Jiří. \textit{Úvod do analogových filtrů} [online]. Praha, 2018 [cit. 2019-03-30]. Dostupné z: \url{https://moodle.fel.cvut.cz/course/view.php?id=1434}. Přednáška. ČVUT FEL. Strana 21, 24, 69, 72.
\renewcommand{\headrulewidth}{0pt}
\fancyhf{}
\bibitem{10}
RAMSDEN, Ed. \textit{An Introduction to Analog Filters}. Sensors Online [online]. 3 Speen Street, Suite 300, Framingham, MA 01701: Questex, 2019, 1/7/2001 [cit. 2019-05-18]. Dostupné z: \url{https://www.sensorsmag.com/components/introduction-to-analog-filters}
\bibitem{11}
\textit{High-pass filtering pre-processing before computing audio features}. Stack Exchange Inc [online]. 2019 [cit. 2019-04-22]. Dostupné z: \url{https://dsp.stackexchange.com/questions/27586/high-pass-filtering-pre-processing-before-computing-audio-features}
\bibitem{12}
MARTINEK, Pravoslav, Petr BOREŠ a Jiří HOSPODKA. \textit{Elektrické filtry}. Praha: Vydavatelství ČVUT, 2003. ISBN 80-01-02765-1. Strana 
29, tabulka 2.12. Strana 74, obrázek 4.17. Strana 141, obrázek 5.43.
\bibitem{13}
SCHAUMANN, Rolf a Mac E. Van VALKENBURG. \textit{Design of Analog Filters}. New York: Oxford University Press, 2001. ISBN 0195118774. Strana 213, obrázek 4-13. Strana 236, obrázek 4-35 a),b). Strana 237, obrázek 4-36 a),b). Strana 608, obrázek 16-2 a),b).
\bibitem{14}
MICHAL, Vratislav. \textit{Vybrané vlastnosti obvodů pracujících v proudovém módu a napěťovém módu} [online]. Brno, 2017 [cit. 2019-03-30]. Dostupné z: \url{https://docplayer.cz/43256146-Vybrane-vlastnosti-obvodu-pracujicich-v-proudovem-modu-a-napetovem-modu.html}. Článek. Brno University of Technology. Strana 5.
\bibitem{15}
SHAKTOUR, Mahmoud. \textit{Nekonvenční obvodové prvky pro návrh příčkových filtrů} [online]. Brno, 2010 [cit. 2019-10-25]. Dostupné z: \url{https://www.vutbr.cz/www_base/zav_prace_soubor_verejne.php?file_id=35975}. Disertační práce. Vysoké učení technické v Brně. Vedoucí práce Dalibor Biolek. Strana 8, obrázek 3-1 (a). Strana 9, obrázek 3-2. Strana 11, obrázek 3--5. Strana 12, obrázek 3-6.
York, 1968 (Vol. 56, no. 3). Článek. IEEE Proc. Strana 1368--1369.
\bibitem{16}
SMITH, K.C., SEDRA, A.S. \textit{The current conveyor: a new circuit building block}. New York, 1968 (Vol. 56, no. 3). Článek. IEEE Proc. Strana 1368--1369.
\bibitem{17}
SMITH, K.C., SEDRA, A.S. \textit{A second generation current conveyor and its application}. New York, 1970 (CT-17). Článek. IEEE Trans. Strana 132--134.
\bibitem{18}
\textit{Transconductance Amplifiers} [online]. 2019 [cit. 2019-03-30]. Dostupné z: \url{https://cz.mouser.com/Semiconductors/Integrated-Circuits-ICs/Amplifier-ICs/Transconductance-Amplifiers/_/N-6j73l?P=1y95od0}
\bibitem{19}
\textit{THD}. In: Wikipedia: the free encyclopedia [online]. San Francisco (CA): Wikimedia Foundation, 2001- [cit. 2019-11-19]. Dostupné z: \url{https://cs.wikipedia.org/wiki/THD}
\bibitem{20}
LM13700: Dual Operational Transconductance Amplifiers With Linearizing Diodes and Buffers. In: \textit{Texas Instruments} [online]. Dallas, Texas: Texas Instruments Incorporated, 2018 [cit. 2019-03-30]. Dostupné z: \url{www.ti.com/lit/ds/symlink/lm13700.pdf} Strana 1. Strana 9, obrázek 16.
\bibitem{21}
\textit{SNR poměr}. In: Optixs.cz [online]. Praha, 2019 [cit. 2019-11-19]. Dostupné z: \url{https://www.optixs.cz/slovnik-17/snr-pomer-70s}
\bibitem{22}
\textit{Signal-to-noise ratio}. In: Wikipedia: the free encyclopedia [online]. San Francisco (CA): Wikimedia Foundation, 2001- [cit. 2019-11-19]. Dostupné z: \url{https://en.wikipedia.org/wiki/Signal-to-noise_ratio}
\bibitem{23}
MOTCHENBACHER, C. D.; CONNELLY, J. A. \textit{Low-noise electronic system design}. [s.l.]: Wiley Interscience, 1993.
\bibitem{24}
\textit{Elektronický šum}. In: Wikipedia: the free encyclopedia [online]. San Francisco (CA): Wikimedia Foundation, 2001- [cit. 2019-11-05]. Dostupné z: \url{https://cs.wikipedia.org/wiki/Elektronick%C3%BD_%C5%A1um#cite_note-noise-1}
\bibitem{25}
GEIGER, Randall L. a Edgar SÂNCHEZ-SINENCIO. \textit{Active Filter Design Using Operational Transconductance Amplifiers: A Tutorial}. IEEE CIRCUITS AND DEVICES MAGAZINE [online]. 1985, 1985 (Březen), 13 [cit. 2019-11-06]. Dostupné z: \url{https://www.ece.uic.edu/~vahe/spring2013/ece412/OTA-structures2.pdf}
\end{thebibliography}