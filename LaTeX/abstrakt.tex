\noindent Cílem práce je navrhout zapojení analogového přeladitelného filtru se zesilovači OTA. Filtr je zvolen typu pásmová propust 4. řádu s Cauerovou aproximací. K realizaci byl použit LM13700 kvůli dostačující šířce pásma (2 MHz) a cenové dostupnosti. Mezní kmitočet byl zvolen v řádu stovek kHz, což umožňuje využití např. pro přenos rozhlasového vysílání v atmosféře. Mezní kmitočet lze měnit vstupním proudem tekoucím do zesilovače až v rozsahu několika dekád. Simulace pásmové propusti s OTA byla realizována v MultiSimu, výhodou zde byla možnost využití bloku LM13700 bez nutnosti modelovat obvod vstupním diferenčním stupněm a proudovými zrcadly. Syntéza filtru z matematického hlediska byla provedena v Maplu s knihovnami Syntfil a PraCAn vyvinutými katedrou teorie obvodů. K praktické realizaci DPS byl využit KiCad z důvodu multiplatformní podpory (Linux, macOS, Windows). \\
\\
\noindent \textbf{Klíčová slova:} \textit{transkonduktance, OTA, OTA-C, analogový filtr, pásmová propust, dolní propust}\\

\noindent The purpose of this thesis is to design a schematics of an analog filter with a variable cut-off frequency using OTA. Filter to design is specified to be a band-pass of fourth-order with Cauer approximation. For the realization was used LM13700 due to its comfortable bandwidth (2 MHz) and price affordability. The cut-off frequency was chosen in the range of hundreds of kHz, which can be used i.e. for transmission of radio broadcasting in the atmosphere. The cut-off frequency can be changed by the input current flowing into the amplifier for up to several decades. Simulation of a bandpass filter with OTA was realized in MultiSim, the advantage was the possibility to use the LM13700 block without the need to model the circuit with a differential input stage and current mirrors. Mathematically, filter synthesis was performed in Maple with the Syntfil and PraCAn libraries developed by the Department of Circuit Theory. For the practical realization of PCB was used KiCad because of its cross-platform support (Linux, macOS, Windows). \\

\noindent \textbf{Klíčová slova:} \textit{transconductance, OTA, OTA-C, analog filter, band-pass, low-pass} \\