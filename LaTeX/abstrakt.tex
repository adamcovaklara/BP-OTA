\noindent Cílem práce je navrhout analogový přeladitelný filtru se zesilovači OTA. Filtr je zvolen typu pásmová propust 4. řádu s Cauerovou aproximací typu C. K realizaci je použit LM13700M kvůli relativně nízkému provoznímu napájecímu proudu (1.3\,mA), rozpětí napájecího napětí (10–-36\,V) a cenové dostupnosti (32.50\,Kč) v porovnání s jinými typy zesilovačů. Mezní kmitočet byl zvolen v řádu stovek kHz, což umožňuje využití např. pro přenos rozhlasového vysílání v atmosféře. Mezní kmitočet lze měnit klidovým stejnosměrným pracovním proudem tekoucím do vstupního diferenčního stupně zesilovače a změnou transkonduktance $g_m$ až v rozsahu 6~dekád. Simulace je realizována v MultiSimu, výhodou tohoto prostředí je možnost využití bloku LM13700M bez nutnosti modelovat obvod vstupním diferenčním stupněm a proudovými zrcadly. Syntéza filtru z matematického hlediska byla provedena v Maplu s použitím knihovny Syntfil vyvinutou katedrou teorie obvodů, FEL ČVUT. K praktické realizaci DPS byl použit program Altium z důvodu přívětivého uživatelského prostředí a množství nabízených možností. \\
\\
\noindent \textbf{Klíčová slova:} \textit{OTA, OTA-C, transkonduktanční zesilovač, transkonduktance, analogový filtr, pásmová propust, dolní propust, Cauerova aproximace}\\

\noindent The purpose of this thesis is to design an analog filter with a variable cut-off frequency using OTA. Designed filter will be a band-pass of fourth-order with Cauer C approximation. For the realization was used LM13700M due to its relatively low operating supply current (1.3\,mA), supply voltage range (10--36\,V) and affordability (1.25\,\euro{} and 1.39\,\$) in comparison with other types of OTA. The cut-off frequency was chosen in the range of hundreds of kHz, which can be used e.g. for transmission of radio broadcasting in the atmosphere. The cut-off frequency can be changed by a DC operating current flowing to the input differential stage of the amplifier and by changing the transconductance value $g_m$ for up to 6~decades. Simulation of a bandpass filter with OTA was realized in MultiSim, the advantage was the possibility to use the LM13700M block without the need to model the circuit with a differential input stage and current mirrors. Mathematically, filter synthesis was performed in Maple with Syntfil library developed by the Department of Circuit Theory, FEL CTU. For the practical realization of PCB was used Altium due to its user-friendly interface and the variety of options available. \\

\noindent \textbf{Key words:} \textit{OTA, OTA-C, operational transconductance amplifier, transconductance, analog filter, band-pass, low-pass, Cauer approximation} \\