\noindent V Altiu byla vytvořena DPS pro DP 4. řádu, PP 2. řádu. Keramický kapacitor byl zvolen 561RTSQ10 (10 pF, jmenovité napětí stejnosměrného proudu 100 V). Symbol a footprint k LM13700 byl stažen z \url{snapeda.com}. Jako zdroj proudu lze použít NPN tranzistor, nebo operační zesilovač. Různé přístupy k řízení OTA vstupním proudem pomocí napěťového zdroje jsou popsány na obrázku \ref{s:DC} (literatura \cite{22}). Obrázek a) je nejjednodušší zapojení, ale toto zapojení je velmi citilivé na malé změny napětí. V zapojení b) je řídící napětí uzemněno, ale $V_c$ je citlivé na změny napětí mezi bází a emitorem tranzistoru a na úbytek napětí na diodě. V zapojení c) je řídící napětí také uzemněno a není závislé na součtu nebo vyrušení napětí na pn přechodech. Zenerova dioda je použita k udržení napětí. Frekvenční odezva OZ se zde neuvažuje, protože máme stejnosměrné napětí. Všechna zapojení jsou velmi citlivá na malé změny $V_c$. K řízení odporu byl použit trimmer o hodnotě odporu 1 MOhm. Zapojení se zdrojem na 0.5 V odpovídá vstupní externí proud 0.5 $\mu A$, který byl použit v simulaci.
\begin{figure}[h]
\centering
\includegraphics[scale=0.5]{current.png}
\caption{Schéma zapojení napěťového zdroje pro vstupní externí proud \label{s:DC}}
\end{figure}
\begin{figure}[h]
\centering
\includegraphics[scale=0.5]{altium.png}
\caption{Schéma obvodu v Altiu}
\end{figure}
\begin{figure}[h]
\centering
\includegraphics[scale=0.5]{altium2.png}
\caption{PCB layout}
\end{figure}