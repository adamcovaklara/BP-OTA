\noindent Cílem práce bylo navrhnout pásmovou propust 4. řádu. Prvním krokem bylo odzkoušení zapojení OTA v Multisimu. Bylo realizováno zapojení filtru typu dolní propust 2. řádu (Sekce \ref{s:DP2}), poté byl kaskádním zapojením filtrů 2. řádu obdržen filtr typu dolní propust 4. řádu (Sekce \ref{s:DP4}). Následně byla zapojením horní a dolní propusti obdržena pásmová propust 2. řádu (Sekce \ref{s:PP2}) a 4. řádu (Sekce \ref{s:PP4}).\\
\\
Pomocí knihovny Syntfil navržena pásmová propust 4. řádu a její zapojení pomocí LC příčkové struktury. Mezikrokem byl převod pásmové propusti na normovanou dolní propust. Pro LC strukturu byly obdrženy odnormované hodnoty prvků. LC struktura byla převedena na zapojení s OTA a byla provedena simulace s vypočtenými prvky. \\
\\
Bude třeba analyzovat výslednou strukturu popsanou přenosy gyrátorů (Sekce \ref{s:MAPLE}) s využitím knihovny Pracan a Maplu. Dalším krokem bude pak praktická realizace a odzkoušení navrhnutého obvodu.