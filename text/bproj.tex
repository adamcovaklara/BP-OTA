%
% Author:  Klára Pacalová
% E-mail:  pacalkla@fel.cvut.cz
% Date:    24.02.2018 11:03
%
\documentclass[twoside]{article}
\usepackage[a4paper]{geometry}
\geometry{verbose,tmargin=2.5cm,bmargin=2cm,lmargin=2cm,rmargin=2cm}
\usepackage{fancyhdr}
\pagestyle{fancy}
\usepackage[utf8]{inputenc}
\usepackage[czech]{babel}
\usepackage{siunitx}
\usepackage{listings}
\usepackage{graphicx} %graphics files inclusion
\usepackage{amsmath} %advanced maths
\usepackage{amssymb} %additional math symbols
% nastavení pisma a češtiny
\usepackage{lmodern}
\usepackage[T1]{fontenc}
\usepackage{float}
\usepackage{xcolor}
\usepackage{authblk}
\graphicspath{{../Images/}}

% odkazy
\usepackage{url}

% vícesloupcové tabulky
\usepackage{multirow}

% automatická konverze schematu
\usepackage{pppfig}

% odkazy a záložky
\usepackage[unicode=true, bookmarks=true,bookmarksnumbered=true,
bookmarksopen=false, breaklinks=false,pdfborder={0 0 0},
pdfpagemode=UseNone,backref=false,colorlinks=true] {hyperref}

% smaz aktualni page layout
\fancyhf{}

% zahlavi
\usepackage{titling}
\fancyhf[HC]{\thetitle}
\fancyhf[HLE,HRO]{\theauthor}
\fancyhf[HRE,HLO]{\today}

 %zapati
\fancyhf[FLE,FRO]{\thepage}

% nastavi novou stranku u sekce
\newcommand{\sekce}[1]{\clearpage \section{#1}}
\newcommand{\nadpis}[1]{\newpage \section*{#1}}

% Maple Output
\definecolor{MapleBlue}{rgb}{0,0,1}
\def\MapleOutput#1{{\begin{center}\begin{math}\color{MapleBlue}{#1}\end{math}\end{center}}}

% titulni strana
\begin{document}

% údaje o autorovi
\title{Analogový přeladitelný filtr se zesilovači OTA}
\author{Klára Pacalová}

% titulni strana
\begin{titlepage}
    \begin{center}
    
        \includegraphics[width=0.85\textwidth]{logo_FEL_cb.jpg}
            
        \vspace*{3cm}
 
        \Huge
        \textbf{Analogový přeladitelný filtr se zesilovači OTA}
 
        \vspace{0.5cm}
 
        \vspace{1.5cm}
 
        \textbf{Klára Pacalová}
 
        \vfill
 
        Bakalářská práce
 
        \vspace{0.8cm}
 		
        \Large
 		Vedoucí práce: Doc. Dr. Ing. Jiří Hospodka\\
    		Katedra teorie obvodů\\
		Fakulta elektrotechnická\\
		České vysoké učení technické v Praze\\
		Technická 2\\
		160 00 Praha 6\\
		Česká republika
		
		\vspace{1.5cm}
		        
		Praha, Prosinec 2019
 
    \end{center}
\end{titlepage}
\nadpis{Abstrakt}
\noindent Cílem práce je navrhout zapojení analogového přeladitelného filtru se zesilovači OTA. Filtr je zvolen typu pásmová propust 4. řádu s Cauerovou aproximací. K realizaci byl použit LM13700 kvůli dostačující šířce pásma (2 MHz) a cenové dostupnosti. Mezní kmitočet byl zvolen v řádu stovek kHz, což umožňuje využití např. pro přenos rozhlasového vysílání v atmosféře, nebo ve \textit{WSN - wireless sensor network}. Mezní kmitočet lze měnit vstupním proudem tekoucím do zesilovače a změnou transkonduktance $g_m$ až v rozsahu několika dekád. Simulace pásmové propusti s OTA byla realizována v MultiSimu, výhodou zde byla možnost využití bloku LM13700 bez nutnosti modelovat obvod vstupním diferenčním stupněm a proudovými zrcadly. Syntéza filtru z matematického hlediska byla provedena v Maplu s knihovnami Syntfil a PraCAn vyvinutými katedrou teorie obvodů. K praktické realizaci DPS byl využit KiCad z důvodu multiplatformní podpory (Linux, macOS, Windows). \\
\\
\noindent \textbf{Klíčová slova:} \textit{transkonduktance, OTA, OTA-C, analogový filtr, pásmová propust, dolní propust}\\

\noindent The purpose of this thesis is to design a schematics of an analog filter with a variable cut-off frequency using OTA. Filter to design is specified to be a band-pass of fourth-order with Cauer approximation. For the realization was used LM13700 due to its comfortable bandwidth (2 MHz) and price affordability. The cut-off frequency was chosen in the range of hundreds of kHz, which can be used i.e. for transmission of radio broadcasting in the atmosphere, or for usage in \textit{WSN - wireless sensor network}. The cut-off frequency can be changed by the input current flowing into the amplifier and by changing the transconductance value $g_m$ for up to several decades. Simulation of a bandpass filter with OTA was realized in MultiSim, the advantage was the possibility to use the LM13700 block without the need to model the circuit with a differential input stage and current mirrors. Mathematically, filter synthesis was performed in Maple with the Syntfil and PraCAn libraries developed by the Department of Circuit Theory. For the practical realization of PCB was used KiCad because of its cross-platform support (Linux, macOS, Windows). \\

\noindent \textbf{Klíčová slova:} \textit{transconductance, OTA, OTA-C, analog filter, band-pass, low-pass} \\
\nadpis{Zkratky}
\vspace*{0.5cm}
ARC \hfill active RC \\ \\
BP \hfill band-pass \\ \\
BS \hfill band-stop \\ \\
BW \hfill band-width \\ \\
CC \hfill current conveyor \\ \\
CCI \hfill current conveyor (first generation) \\ \\
CCII \hfill current conveyor (second generation) \\ \\
CMMR \hfill common mode rejection ratio \\ \\
CMOS \hfill complementary metal–oxide–semiconductor \\ \\
DIDO \hfill differential-input, differential-output \\ \\
DP \hfill dolní propust \\ \\
DPS \hfill deska plošných spojů \\ \\
EKG \hfill elektrokardiogram \\ \\
ESD \hfill elektrostatický výboj \\ \\
GBP \hfill gain bandwidth product \\ \\
GIC \hfill general impedance converter \\ \\
HD \hfill harmonic distortion \\ \\
HP \hfill horní propust \\ \\
HP \hfill high-pass \\ \\
IC \hfill integrated circuit \\ \\
LP \hfill low-pass \\ \\
PAN \hfill personal area network \\ \\
PDIP \hfill plastic dual-in-line \\ \\
OTA \hfill operational transconductance amplifier \\ \\
OZ \hfill operační zesilovač \\ \\
PCB \hfill printed circuit board \\ \\
PP \hfill pásmová propust \\ \\
PZ \hfill pásmová zádrž \\ \\
RF \hfill radio frequency \\ \\
SaH \hfill sample and hold \\ \\
SISO \hfill single-input, single-output \\ \\
SMD \hfill surface mount device \\ \\
SNR \hfill signal-to-noise ratio \\ \\
SOIC \hfill small outline integrated circuit \\ \\
SR \hfill slew rate \\ \\
THD \hfill total harmonic distortion \\ \\
UGBW \hfill unity gain bandwidth \\ \\
WSN \hfill wireless sensor network \\ \\

% seznamy
\newpage
\tableofcontents
\newpage
\listoffigures
\newpage
\listoftables

\sekce{Úvod}
Mezní kmitočet navrhovaného filtru byl zvolen v řádu stovek kHz, což umožňuje využití např. pro přenos rozhlasového vysílání v atmosféře, nebo je po zvýšení mezního kmitočtu, aby odpovídal standardu ZigBee, možné využití ve \textit{WSN} (\textit{Wireless Sensor Network}) pro monitorování dat z environmentálních senzorů v dané lokalitě. \\
\textit{WSN} běžně používá standard ZigBee s frekvencemi 868 MHz, 902–928 MHz a 2,4 GHz. ZigBee patří do skupiny bezdrátových sítí \textit{PAN (Personal Area Networks)} a je určena pro spojení nízkovýkonových zařízení v těchto sítích na malé vzdálenosti (do 75 metrů). Umožňuje komunikaci i na větší vzdálenosti bez přímé radiové viditelnosti jednotlivých zařízení. Do této skupiny sítí patří i velmi rozšířený IEEE 802.15.1 – Bluetooth (literatura \cite{1}).\\
Kmitočet v řádu stovek kHz odpovídá dlouhým a středním vlnám. Dlouhé vlny (30 -- 300~kHz, čemuž odpovídá délka vlny 1 -- 10~km) obtékají nerovnosti a jdou za obzor bez nutnosti odrazu. Dnes na dlouhých vlnách vysílá jen několik národních rozhlasových vysílačů velkých států a pásmo se hlavně využívá pro takové účely, kde je na prvním místě spolehlivost a výhody pozemní vlny. To jsou například frekvenční a časové standardy (DCF77 --  rádiová stanice vysílající dlouhovlnný tzv. frankfurtský časový signál), radiomajáky, případně i komunikace s ponorkami (literatura \cite{2}). Střední vlny (525 -- 1705 khz, což odpovídá vlnovým délkám 186 -- 577 m) mají menší dosah a často u nich dochází k jednomu odrazu od atmosféry. Lépe se ohýbají za přírodními překážkami a jsou vhodné pro vysílání v okruhu stovek kilometrů (literatura \cite{3}). \\
Mezní kmitočet a zesílení lze pomocí vstupního proudu tekoucího do zesilovače a pomocí změny transkonduktance $g_m$ měnit až v rozsahu několika dekád. Při návrhu filtru s OTA není použita zpětná vazba -- její absence je výhodná z hlediska stability a kmitočtového rozsahu.
\subsection{Analogové filtry}
Filtry jsou určeny k potlačení nebo zvýraznění určité části kmitočtového spektra signálu. Jsou to obvody s kmitočtově závislou přenosovou funkcí (pro napěťový přenos $H_s(j \omega) = U_{out}(j \omega)/U_{in}(j \omega)$). Základní rozdělení je na dolní propust (DP, anglicky \textit{low-pass} - LP), horní propust (HP, \textit{high-pass} - HP), pásmovou propust (PP, \textit{band-pass} - BP) a pásmovou zádrž (PZ, \textit{band-stop} - BS). \\
\begin{figure}[h]
\centering
\includegraphics[scale=0.55]{tolerancnischemata.png}
\caption[Toleranční schéma dolní propusti (DP), horní propusti (HP), pásmové propusti (PP) a pásmové zádrže]{Toleranční schéma pro a) dolní propust (DP), b) horní propust (HP), c) pásmovou propust (PP) a d) pásmovou zádrž (PZ)\cite{4}}
\end{figure}
\noindent Dolní propust nepropouští na výstup vstupní signál nad frekvencí $f_s$, signál v propustném pásmu zůstává beze změny nebo zesílený. Základní pasivní dvojbranné zapojení je ke vstupu sériově zapojený rezistor a k této větvi paralelně kapacitor. Tento RC člen (integrační článek) se zvyšující se frekvencí snižuje svou vstupní impedanci. Přenosová funkce má nulu v nekonečnu a pól v levé polorovině s-roviny. Ideální integrátor má pól v nule. \\
Obecná přenosová funkce filtru typu dolní propust je
\begin{equation}
H(j\omega) = \frac{H_0}{\prod_{i=1}^{\frac{n}{2}} (1 + a_i s + b_i s^2)},
\end{equation}
kde $n$ je řád filtru.\\
Dolní propust druhého řádu má přenos v nekonečnu nulový $H_{\infty} = 0$. Přenosová funkce je
\begin{equation}
H(j\omega) = \frac{H_0 \omega_c ^2}{(j\omega)^2 + \frac{\omega _c}{Q}(j\omega) + \omega _c ^2}.
\end{equation}
Horní propust nepropouští signály o nízkých frekvencích. Nejjednodušší zapojení je RC člen (derivační článek), kdy kapacitor je zapojen sériově se zdrojem a k této větvi paralelně rezistor. Pro toto zapojení reaktance kapacitoru se zvyšující se frekvencí klesá. Přenosová funkce ideálního derivátoru má pól v nekonečnu a nulu v nule. Horní propust má nulu v nule a pól v levé polorovině s-roviny.\\
Obecná přenosová funkce filtru typu horní propust je
\begin{equation}
H(j\omega) = \frac{H_{\infty}}{\prod_{i=1}^{\frac{n}{2}} (1 + \frac{a_i}{s} + \frac{b_i}{s^2})},
\end{equation}
kde $n$ je řád filtru.\\
Horní propust druhého řádu (s každým řádem s pokles mění o 20 dB/dek) má přenos v nule nulový $H_{0} = 0$. Přenosová funkce je
\begin{equation}
H(j\omega) = \frac{H_{\infty} (j\omega) ^2}{(j\omega)^2 + \frac{\omega _c}{Q}(j\omega) + \omega _c ^2}.
\end{equation}
\\
Pásmová propust propouští pásmo určené dvěma kmitočty. Pasivní pásmové propusti nedosahují účinnosti větší než 1. Jsou složeny z integračního článku (RC - dolní propust) a derivačního článku (CR - horní propust).\\
\begin{figure}[h]
\centering
\includegraphics[scale=0.6]{bpbs.png}
\caption[A) Násobení přenosů - pásmová propust, B) Sčítání přenosů - pásmová zádrž]{A) Násobení přenosů - pásmová propust, B) Sčítání přenosů - pásmová zádrž \cite{5}}
\end{figure}
\noindent Pásmová propust má přenos v nule i nekonečnu nulový $H_{0} = H_{\infty} = 0$. Přenosová funkce je
\begin{equation}
H(j\omega) = \frac{H_{B} \frac{\omega _c}{Q} (j\omega) }{(j\omega)^2 + \frac{\omega _c}{Q}(j\omega) + \omega _c ^2}.
\end{equation}
\\
Pásmová zádrž nepropouští kmitočty pásma definovaného dvěma kmitočty. Pasivní zapojení je složeno ze dvou rezistorů a kapacitorů. Má vždy ztrátový přenos. \\
\noindent Filtry se používají k redukci nežádoucích frekvencí např. pro efektivní reprodukci zvuku reproduktory, k redukci okolního rušení - vysílače blokují harmonické frekvence, které interferují. Také v obvodech rekonstrukce signálů u D/A převodníků, k předvzorkování u A/D převodníku nebo jako anti-aliasing filtry.\\\\
Podle rozložení nul a pólů jmenovatele rozlišujeme různé aproximace. Koeficienty filtru $a_i, b_i$ určují zesílení v propustném pásmu. Činitel jakosti je definován jako $Q = \sqrt{b_i}/a_i$. Čím větší $Q$ je obdrženo, tím spíš bude filtr nestabilní.
\begin{figure}[h]
\centering
\includegraphics[scale=0.3]{LGA98.png}
\caption[Typy aproximací (DP)]{Typy aproximací (DP)\cite{6}}
\end{figure}
\subsection{Butterworthova aproximace}
Podle literatury \cite{4} má Butterworthova aproximace maximálně plochou amplitudovou charakteristiku v propustném pásmu. Frekvenční charakteristika má sklon daný počtem pólů a pro její posouzení je užíváno skupinové zpoždění (derivace fáze podle frekvence). Pro Butterworthovu aproximaci je skupinové zpoždění nezvlněné v propustném pásmu. Přechodová charakteristika má mírný překmit, zvyšující se s řádem filtru. Zesílení $G(\omega)$ je kmitočtově závislé a odpovídá absolutní hodnotě přenosové funkce $H(j\omega)$.
\begin{equation}
G(\omega) = |H(j\omega)| = \frac{1}{\sqrt{1 + \epsilon ^2 \frac{\omega}{\omega _c}^{2n}}},
\end{equation}
kde $\epsilon$ je poměrné zvlnění kmitočtové charakteristiky v propustném pásmu (\textit{faktor zvlnění}), $n$ je řád filtru a $\omega _c$ mezní kmitočet (nastává při útlumu -3 dB). Pro $\omega _c = 1$ je faktor zvlnění $\epsilon = 1$. 
\subsection{Čebyševova aproximace}
Čebyševova aproximace má strmější pokles, což vede k užití nižšího řádu filtru. Zato má ale zvlněnou frekvenční charakteristiku v propustném pásmu. 
\subsubsection{Typ I}
Vyjádření modulové charakteristiky pro tuto aproximaci je dáno jako
\begin{align}
G(\omega) = |H(j\omega)| = \frac{1}{\sqrt{1 + \epsilon ^2 T_n ^2 \frac{\omega}{\omega _c}^{2n}}},
\end{align}
kde $T_n$ je Čebyševův polynom, $\epsilon$ je poměrné zvlnění, $n$ je řád filtru a $\omega _c$ mezní kmitočet. Čebyševův polynom je definován vztahem $2 \omega ^2 - 1$ pro $n = 2$. Obecně jsou to kořeny Chebyshevových diferenciálních rovnic
\begin{align}
(1 - x^2)y" - xy' + n^2y &= 0\\
(1 - x^2)y" - 3xy' + n(n+2)y &= 0.
\end{align}
\subsubsection{Typ II}
Typ II je nazýván také jako inverzní Čebeševova aproximace. V praxi není příliš používaný, jelikož nemá tak rychlý pokles jako typ I a k jeho realizaci je třeba více prvků. Nemá zvlnění v propustném pásmu, zato v zádržném ano. Zesílení je definováno jako
\begin{equation}
G(\omega, \omega _c) = \frac{1}{\sqrt{1 + \frac{1}{\epsilon ^2 T_n ^2 \frac{\omega _c}{\omega}^{2n}}}},
\end{equation}
kde $T_n$ je Čebyševův polynom, $\epsilon$ je poměrné zvlnění, $n$ je řád filtru a $\omega _c$ mezní kmitočet.
\subsection{Besselova aproximace}
Besselova aproximace se používá v telekomunikační technice v případech, kdy je požadováno zachování tvaru signálu. Amplitudová charakteristika v nepropustném pásmu je velmi plochá. Koeficienty polynomu jsou zvoleny tak, aby fázová charakteristika v pásmu okolo kritické frekvence byla maximálně lineární. Nevýhodou je poměrně malá strmost modulové charakteristiky. Ta je pro Besselovu aproximaci dána vztahem
\begin{equation}
G(\omega) = |H(j\omega)| = \frac{\Theta _n(0)}{\Theta _n(\frac{j\omega}{\omega _c})},
\end{equation}
kde $\Phi _n$ je Besselův polynom a $\omega _c$ mezní kmitočet. Besselův polynom je definován součtem řady (Grosswald 1978, Berg 2000)
\begin{equation}
\Theta _n (x) = x^n y_n (\frac{1}{x}) = \sum_{k=0}^{n}\frac{(n+k)!}{(n-k)!k!}\frac{x^{n-k}}{2^k}.
\end{equation}
Pro filtr druhého řádu platí
\begin{equation}
G(\omega) = |H(j\omega)| = \frac{3}{\sqrt{\omega ^4 + 3\omega ^2 + 9}}.
\end{equation}
\subsection{Cauerova (eliptická) aproximace}
\noindent Cauerova aproximace (eliptická) má nejstrmější pokles, při jejím užití jsou voleny nižší řády filtru. Pokud se zvlnění v zádržném pásmu blíží nule, filtr se stává Čebyševovým (výše zmíněný - typ I). Opačně je tomu v propustném pásmu - přiblížením k nule se filtr stává inverzním Čebyševovým (typ II).  Pokud se obě hodnoty zvlnění blíží k nule, filtr se stává Butterworthovým. Kmitočtová charakteristika je dána vztahem
\begin{equation}
G(\omega) = |H(j\omega)| = \frac{1}{\sqrt{1 + \epsilon ^2 R_n ^2(\zeta, \frac{\omega}{\omega _c})}},
\end{equation}
kde $\epsilon$ je faktor zvlnění, $R_n$ eliptická racionální funkce n-tého řádu, $\zeta$ selektivní faktor a $\omega _c$ mezní kmitočet. Pokud pro selektivní faktor platí $\zeta \rightarrow \infty$, filtr se stává Čebyševovým (typ I).\\
Protože se, podobně jako u Čebyševovy aproximace, liší odvození pro liché a sudé stupně, jsou pro ně různé postupy. Pro lichý stupeň existuje pouze jedna varianta, pro sudý tři varianty (A, B, C), které se liší průběhem aproximační funkce - viz literatura \cite{7}.
\subsubsection{Cauerova (eliptická) aproximace typu A}
Má stejný počet pólů a nul aproximující funkce. Je realizována jako LC filtr (Sekce \ref{s:LC}) pouze s vázanými induktory.
\subsubsection{Cauerova (eliptická) aproximace typu B}
Jedná se o posun útlumového pólu z konečného kmitočtu k nekonečnu, tedy dále od propustného pásma. Tato úprava vede ke snížení strmosti přechodu od propustného k nepropustnému pásmu. Je to obdoba postupu u inverzní Čebyševovy aproximace.
\subsubsection{Cauerova (eliptická) aproximace typu C}
Vhodnou transformací, která vede na nulovou hodnotu přenosu v nulovém kmitočtu, získáme navíc proti variantám A, B i shodné zakončovací odpory v případě LC realizace (Sekce \ref{s:LC}). Je to obdoba postupu u Čebyševovy aproximace.
\subsection{Srovnání typů aproximací}
Podle literatury \cite{7} není z hlediska zápisu přenosové funkce rozdíl mezi Butterworthovou a Čebyševovou aproximací, přestože jedna má v propustném pásmu hladký a druhá zvlněný průběh. Přenosová funkce má v čitateli konstantu a ve jmenovateli polynom (odtud společné označení polynomiální aproximace).\\
Oproti tomu volba průběhu v nepropustném pásmu tvar přenosové funkce mění. Pokud je průběh monotónní (Butterworthova, Čebyševova aproximace), jedná se o podíl konstanty a polynomu. Je-li průběh v nepropustném pásmu zvlněný, tvoří přenosovou funkci podíl dvou polynomů. Pro běžné aproximace (Cauerova, inverzní Čebyševova) je v čitateli sudý polynom ve tvaru $\prod _{i} (p^2 + \omega_i^2)$.\\
Volbou kombinace hladkého a zvlěného průběhu v propustném a nepropustném pásmu získáme různé vlastnosti.\\
 \begin{table}[h]
 \caption[Přehled aproximací podle tvaru aproximační funkce v propustném i nepropustném pásmu]{\label{tab:Přehled aproximací podle tvaru aproximační funkce v propustném i nepropustném pásmu}Přehled aproximací podle tvaru aproximační funkce v propustném i nepropustném pásmu \cite{7}}
  \end{table}
\begin{center}
\begin{table}[h]
\centering
  \begin{tabular}{ | c | c | c | }
    \hline
    Propustné pásmo & Nepropustné pásmo & Příklad aproximace \\ \hline
    hladká & hladká & Butterworthova \\ \hline
    zvlněná & hladká & Čebyševova \\ \hline
    hladká & zvlněná & inverzní Čebyševova \\ \hline
    zvlněná & zvlněná & Cauerova \\ \hline
  \end{tabular}
   \end{table}
   \end{center}
\sekce{Transkonduktanční zesilovače (OTA)}
Podle Schaumanna a Valkenburga \cite{13} se v~telekomunikacích používají filtry v rozsahu kmitočtů desítek až stovek MHz, v bezdrátové komunikaci až v řádu GHz. Běžné RC filtry by neměly být užívány ve frekvenčním rozsahu nad 5--10 $\%$ $\omega _0$ ($\omega _0 = f_m \cdot 2 \pi$, kde $f_m$ je mezní kmitočet) --- tedy v tomto rozsahu používaném v telekomunikačních technologiích nemají předvídatelné průběhy. Krom toho ve spínačích CMOS, kde rezistory běžně nejsou dostupné, jsou potřeba zesilovače s velkou šířkou pásma a zároveň vysokým zesílením. Dodržení těchto požadavků je náročné a drahé. Dalším extrémem pro analogové integrované filtry jsou telefonní linky, kde jsou kmitočtové rozsahy sice nízké, ale je požadována nízká cena a vysoká přesnost.\\
\\
Pro nízké frekvence se ke splnění těchto požadavků používají obvody se spínanými kapacitory (SC). Přepínaný kapacitor se chová jako rezistor, tudíž časová konstanta RC je definována poměrem kapacitorů a hodinovou (CLK) frekvencí, se kterou jsou přepínány. Pro vysokofrekvenční aplikace (v řádu GHz) se používají \newline MOSFET-C filtry.\\
\\
Další z možných prvků, které jsou dostupné jak pro nízkofrekvenční aplikace, tak pro kmitočtový rozsah stovek megahertz, jsou transkonduktanční zesilovače. Jejich kmitočtové vlastnosti umožňují využití při konstrukci ARC (\textit{Active RC}) filtrů v pracovním kmitočtovém pásmu do cca 10\,MHz a se speciálně konstruovanými OTA až do 100\,MHz. Kmitočet dominantního pólu je běžně v oblasti stovek kHz až jednotek MHz (Martinek, Boreš, Hospodka \cite{12}).\\
\\
Transkonduktanční zesilovače (označují se též jako OTA (\textit{Operational Transconductance Amplifiers}) jsou napětím řízené zesilovače s proudovým výstupem --- zdroje proudu
\begin{equation}
i_{out} = g_m(u_+ - u_-),
\end{equation}
kde $u_+$ a $u_-$ jsou napětí invertujícího a neinvertujícího vstupu. Transkonduktance je řízena klidovým stejnosměrným pracovním proudem $I_{ABC}$. Ideální OTA má kmitočtově nezávislou transkonduktanci $g_m$ (na rozdíl od reálného, který je kmitočtově závislý). Ideální OTA má také nekonečnou vstupní a výstupní impedanci. S~ohledem na proudový (t.j. vysokoimpedanční) výstup OTA je výraznější vliv výstupní impedance. Ta má jak odporovou, tak kapacitní složku a musí být při návrhu ARC obvodů (sekce \ref{s:ARC}) respektována. Vstupní impedance, s ohledem na to, že OTA bývá implementován v CMOS technologii, má převážně kapacitní charakter (Martinek, Boreš, Hospodka \cite{12}).
\begin{figure}[h]
\centering
\includegraphics[scale=0.7]{image7.png}
\caption[OTA --- schematické značky]{OTA --- schematické značky \cite{14}}
\end{figure}
\begin{figure}[h]
\centering
\includegraphics[scale=0.55]{gmrc.png}
\caption[Linearizovaný model reálného OTA]{Linearizovaný model reálného OTA \cite{9}}
\end{figure}
\noindent Připojením zátěže $R_z$ na výstup bylo získáno napětí naprázdno
\begin{equation}\label{s:vzt}
u_{out} = R_zg_m(u_+ - u_-) = G_0(u_+ - u_-),
\end{equation}
kde $G_0$ je zesílení. Ze vztahu \ref{s:vzt} plyne, že zesílení je konečné a mezi vstupy je nenulové napětí. \\
\noindent Po připojení kondenzátoru jako zátěže byl získán bezeztrátový integrátor s přenosem
\begin{equation}
H(p) = \frac{v_2}{v_1} = \frac{g_m}{pC}
\end{equation}
\noindent a napětím na výstupu
\begin{equation}
v_0(t) = \frac{1}{C}\int i(t)dt = \frac{1}{C}\int g_mv_1(t)dt.
\end{equation}
\begin{figure}[h]
\centering
\includegraphics[scale=0.45]{otaintegrator.png}
\caption[OTA-C]{OTA-C \cite{9} \label{s:GM-C}}
\end{figure}
\noindent Toto zapojení integrátoru s uzemněným kondenzátorem se označuje jako OTA-C.\\
\\
Ztrátový integrátor lze utvořit sériovým zapojením dalšího OTA jako odporu se zápornou zpětnou vazbou. Rozdíl mezi ideálním a ztrátovým integrátorem lze pozorovat v modulové charakteristice --- pro ztrátový je konstantní a pak teprve lineárně klesá se sklonem 20\,dB/dek.\\
Po doplnění ztrátové vodivosti, kterou zde simuluje druhý zesilovač, paralelně k integrační kapacitě, byl obdržen vztah pro výstupní napětí
\begin{equation}
v_0(t) = \frac{g_{m1}}{sC + g_{m2}}(v_1^+ - v_{1}^-).\label{s:OTA-INT1}
\end{equation}
\begin{figure}[h]
\centering
\includegraphics[scale=0.45]{damp.png}
\caption[Ztrátový OTA-C]{Ztrátový OTA-C \cite{9} \label{s:OTA-INT}}
\end{figure}
\subsection{Proudový konvejor druhé generace s OTA}
Podle Shaktoura \cite{15} je jeden z nejzákladnějších bloků v oblasti analogových obvodů v~proudovém módu proudový konvejor CC (\textit{Current Conveyor}). Princip CC první generace byl popsán v roce 1968 (K. C. Smith, A. S. Sedra \cite{16}). CCI byl následně nahrazen univerzálnější druhou generací v roce 1970 (CCII)\cite{17}. Obvody s CC se používaly především v zapojeních s bipolárními tranzistory kvůli jejich vysoké transkonduktanci (v~porovnání s CMOS). Jsou to operační zesilovače s proudovou zpětnou vazbou (např. MAX477, MAX4112). Proudové konvejory (\textit{Current Conveyors}) jsou používány ve~vysokofrekvenčních obvodech, kde je problematické použití běžných operačních zesilovačů, protože jsou limitovány násobkem šířky pásma a zesílení GBP (\textit{Gain-Bandwidth Product}). Je to struktura s třemi vstupy.
\begin{figure}[h]
\centering
\includegraphics[scale=0.3]{ccii.png}
\caption[CCII symbol]{CCII symbol \cite{15}}
\end{figure}
\noindent Proudovým konvejorem lze také jednoduše realizovat integrátor. Pro výstupní napětí $u_0$ obvodu a z něj odvozenou přenosovou funkci platí
\begin{align}
u_0(t) &= \frac{1}{C}\int i_c(t)dt = \frac{1}{RC}\int u_{in}(t)dt, \\
H(p) &= \frac{U_2}{U_1} = \frac{1}{pCR}.
\end{align}
\noindent Konvejorový integrátor pracuje jako invertující nebo neinvertující.
\begin{figure}[h]
\centering
\includegraphics[scale=0.062]{ccii_int1.png}
\caption[Integrátor s CCII a OTA]{Integrátor s CCII a OTA \cite{12}}
\end{figure}
\noindent Vstupní impedance na vstupu Y je nekonečná (tedy proud tekoucí skrz Y je nulový) a impedance na vstupu X je nulová ($R_Y = \infty, I_Y = 0, R_X = 0$). Napětí na vstupu X je ekvivalentní k napětí na vstupu Y ($V_X = V_Y$). Proud procházející vstupem X je ekvivalentní k~proudu vstupem Z ($I_Z = I_X$). Výstupní impedance vstupu Z je nekonečná ($R_Z = \infty$).
Charakteristika ideálního CC je reprezentována maticí
\begin{equation}
\begin{bmatrix}
I_Y \\ V_X \\ I_Z
\end{bmatrix}
=
\begin{bmatrix}
0 & 0 & 0 \\
1 & 0 & 0 \\
0 & \pm 1 & 0 
\end{bmatrix}
\begin{bmatrix}
V_Y \\
I_X \\
V_Z
\end{bmatrix}.
\end{equation}
\begin{figure}[h]
\centering
\includegraphics[scale=0.45]{cciiota.png}
\caption[CCII s $\pm$ výstupem založený na OTA]{CCII s $\pm$ výstupem založený na OTA \cite{15}}
\end{figure}
\newline
V analogových IC je preferováno diferenční zpracování signálů, protože redukuje zkreslení a šum (diferenční stupeň vyruší kladné a záporné výchylky napětí/proudu např. na zdroji a také vyruší nelinearity způsobené zesilovačem).\\
\\
Využitím zapojení na obrázku \ref{s:OTA} a principů CCII lze získat modifikace klasického transkonduktančního zesilovače (s rozdílovým stupněm na vstupu a jedním výstupem). Obdržené atypické struktury obsahují jeden vstup a jeden výstup a také dva rozdílové stupně (na vstupu i výstupu). Provodení diferenčního stupně na výstupu je znázorněno na obrázku \ref{s:CMOS}. Takovéto zapojení funguje jako dobrý sledovač napětí, ale zato má menší šířku pásma. Také má menší transkonduktanci, protože každou polovinou diferenčního obvodu teče jen polovina klidového stejnosměrného pracovního proudu. Problémem je také relativně nízké stejnosměrné zesílení, proto se v~praxi se~zapojením s OTA nepoužívá. \\
\begin{figure}[h]
\centering
\includegraphics[scale=0.4]{diff.png}
\caption[Základní CMOS transkonduktance a) jeden výstup b) diferenční výstup]{Základní CMOS transkonduktance a) jeden výstup b) diferenční výstup \cite{13} \label{s:CMOS}}
\end{figure}
\begin{figure}[h]
\centering
\includegraphics[scale=0.6]{siso.png}
\caption[Single input single output OTA (SISO) založený na CCII]{Single input single output OTA (SISO) založený na CCII \cite{15}}
\end{figure}
\begin{figure}[h]
\centering
\includegraphics[scale=0.55]{dido.png}
\caption[\textit{Fully differential} OTA (DIDO) založený na CCII a napěťovém bufferu]{\textit{Fully differential} OTA (DIDO) založený na CCII a napěťovém bufferu \cite{15}}
\end{figure}
\newpage
\subsection{IC s OTA}
OTA bývá nejčastěji implementován v unipolární monolitické technologii (CMOS) a v případě, že se jedná o~reálný prvek, má převážně kapacitní charakter (vstupní impedance se stává kapacitou). Integrované obvody se vyrábí buď s jedním nebo dvěma zesilovači v pouzdře. Varianty s jedním operačním zesilovačem jsou např. OPA615, OPA860 a novější OPA861. Všechny součástky s jedním OTA mají velkou šířku pásma (v řádech stovek MHz), cenově vychází na 85--290\,Kč. IC s dvěma zesilovači v pouzdře mají užší šířku pásma (2\,MHz), menší rychlost přeběhu (50\,V/$\upmu$s), mnohem menší výstupní proud (650\,$\upmu$A) i offset vstupního napětí a operují při cca 4x nižších proudech. Cenové rozpětí pro pouzdro čipu s 2 OTA je 35--65\,Kč. Ceny v tabulce jsou uváděny pro jednu součástku k 10. 11. 2019. Pokud se vyrábí více verzí součástky, je vždy pro přehlednost uvedena pouze jedna. GBP (\textit{Gain-Bandwidth Product}) v tabulce je získán vynásobením hodnoty kmitočtu dominantního pólu celkovým zesílením. Například, podle Michala \cite{14}, je-li navrhován zesilovač se zesílením G, je hodnota mezního kmitočtu pro pokles 3\,dB dána vztahem
\begin{equation}
f_m = \frac{GBP}{G}.
\end{equation}
\noindent Pro jednotkové zesílení je hodnota mezního kmitočtu rovna GBP. Někdy se místo GBP udává parametr UGB (\textit{Unity Gain Bandwidth} --- šířka pásma při jednotkovém zesílení), který má shodný význam, nebo tranzitní kmitočet $f_T$. Tranzitní kmitočet $f_T$ je kmitočet, na kterém klesne zesílení OZ na 0\,dB, tj. na kterém přestává OZ zesilovat. Šířka pásma s útlumem 3\,dB je funkcí klidového stejnosměrného pracovního proudu a pro proudy pod 100 $\upmu$A je platí přibližně 3\,dB\ BW = $3 \cdot 10^{11} I_{ABC}$.
\renewcommand{\arraystretch}{1.5}
\begin{table}[h]
\scalebox{0.75}{%
  \begin{tabular}{ | c | >{\centering\arraybackslash}p{0.8cm}| >{\centering\arraybackslash}p{1.0cm} | >{\centering\arraybackslash}p{1.25cm} | >{\centering\arraybackslash}p{1cm} | >{\centering\arraybackslash}p{1.0cm} | >{\centering\arraybackslash}p{1.2cm} | >{\centering\arraybackslash}p{1.4cm} | >{\centering\arraybackslash}p{1.2cm} |>{\centering\arraybackslash}p{0.9cm} |>{\centering\arraybackslash}p{0.85cm} |}
    \hline
      & GBP [MHz] & SR [kV/$\upmu$s] & Výstupní proud na kanál [mA] & $I_b$ --- vstupní klidový proud [$\upmu$A] & $V_{os}$ --- vstupní napěťová nesymetrie [mV] & Provozní napájecí proud [mA] & Minimální transkonduktance [mA/V] & Napájecí napětí [V] &  Šířka pásma 3 dB [MHz] & Cena [Kč] \\ \hline
    OPA615ID & 710 & 2.5 & 5 & 3 & 40 & 13 & 65 & 8--12.4 & 710 & 285.22\\ \hline
    OPA860ID & 470 & 3.5 & 15 & 5 & 12 & 11.2 & 80 & 5--13 & 470 & 168.22\\ \hline
    OPA861ID & 400 & 0.9 & 15 & 1 & 12 & 5.4 & 65 & 4--12.6 & 80 & 82.68\\ \hline
     LT1228CS8\#PBF & 100 & 0.5 & 65 & 0.4 & 3 & 9  & 0.75 & 4--36 & 80 &  210.34\\
    \hline
  \end{tabular}}
  \caption[Porovnání integrovaných obvodů s jedním OTA]{\label{tab:Porovnání IC s jedním OTA}Porovnání IC s jedním OTA \cite{18}}
  \end{table}
\begin{center}
\begin{table}[h]
\scalebox{0.75}{%
  \begin{tabular}{ | c | >{\centering\arraybackslash}p{0.8cm}| >{\centering\arraybackslash}p{0.85cm} | >{\centering\arraybackslash}p{1.25cm} | >{\centering\arraybackslash}p{1.0cm} | >{\centering\arraybackslash}p{1.0cm} | >{\centering\arraybackslash}p{1.25cm} | >{\centering\arraybackslash}p{1.45cm} | >{\centering\arraybackslash}p{1.25cm} |>{\centering\arraybackslash}p{1.0cm} |>{\centering\arraybackslash}p{0.75cm} |}
    \hline
      & GBP [MHz] & SR [V/$\upmu$s] & Výstupní proud na kanál [$\upmu$A] & $I_b$ --- vstupní klidový proud [$\upmu$A] & $I_{os}$ --- vstupní proudová nesymetrie [mV] & Provozní napájecí proud [mA] & Minimální transkonduktance [$\upmu$S] & Napájecí napětí [V] & CMMR [dB] & Cena [Kč] \\ \hline
    LM13700M & 2 & 50 & 650 & 5 & 4 & 1.3 & 6700 & 10--36 & 80--110 & 32.50 \\ \hline
    NE5517DG & 2 & 50 & 650 & 5 & 5 & 2.6 & 5400 & 4--44 & 80 & 43.42 \\ \hline
    AU5517DR2G & 2 & 50 & 650 & 5 & 5 & 2.6 & 5400 & 4--44 & 80 & 64.48 \\ \hline
    NJM13600M & 2 & 50 & 650 & 5 & 5 & 2.6 & 6700 & 36 & 80 & 36.92 \\ \hline
    NJM13700M & 2 & 50 & 650  & 5 & 4 & 2.6 & 6700 & 36 & 80 & 37.70 \\ \hline
  \end{tabular}}
  \caption[Porovnání integrovaných obvodů se dvěma OTA]{\label{tab:Porovnání IC se dvěma OTA}Porovnání IC se dvěma OTA \cite{18}}
  \end{table}
\end{center}
\noindent Pro realizaci přeladitelného filtru byl zvolen LM13700M kvůli provoznímu napájecímu proudu (1.3\,mA), rozpětí napájecího napětí (10–-36\,V) a cenové dostupnosti (32.50\,Kč). Součástky NJM13700, NJM13600 a LM13600 se dají zakoupit, ale mají plánované vyřazení z výroby kvůli zastarání. \\
\\
LM13700 má linearizující diody a buffery s diferenciálním vstupem a push-pull výstupem. Dva zesilovače na čipu mají stejné napájení, ale jinak fungují nezávisle na sobě. Linearizační diody snižují zkreslení, a tak umožňují vyšší hodnoty amplitudy vstupního signálu. Výsledkem je zesílení odstupu signál-šum o 10\,dB (0.5\,\% THD). Když analogový signál prochází nelineárním zařízením, k původnímu signálu jsou přimíchány další frekvence. THD je způsobem posouzení rozsahu zkreslení.\\
\\
Výpočet THD vychází z rozkladu periodického signálu na harmonické složky pomocí Fourierovy řady v amplitudově-fázovém zápisu. Nejčastěji se definuje jako podíl součtu výkonů všech harmonických frekvencí nad základní harmonickou k základní harmonické. Čím nižší je THD, tím věrnější je signál zachycený nebo předávaný pomocí mikrofonu, reproduktoru nebo zesilovače \cite{19}. \\
\\
\begin{equation}
THD = \frac{\sum{výkon\ vyšších\ harmonických}}{výkon\ základní\ harmonické} = \frac{P_2 + P_3 + \cdots + P_n}{P_1} \cdot 100 \%
\end{equation}
Vysokoimpedanční buffery doplňují dynamický rozsah zesilovače. Výstupní buffery LM13700 se liší od předchozí, již nevyráběné, řady 13600 v tom, že jejich vstupní klidové proudy $I_b$ (a tedy jejich výstupní stejnosměrné úrovně) jsou nezávislé na klidovém stejnosměrném pracovním proudu $I_{ABC}$. To má za následek např. vyšší výkon ve zvukových aplikacích než s LM13600. LM13700 je přeladitelný přes 6 dekád a má výbornou linearitu transkonduktance $g_m$. LM13700 má na výstupu vysokou hodnotu odstupu signál od šumu \cite{20}.\\
\\
Porovnání ceny typů LM13700 je uvedeno v tabulce níže. Parametry jsou stejné, typy se od sebe liší počtem kusů v balení a typem pouzdra (SOIC/PDIP).
\begin{table}[h]
\centering
  \begin{tabular}{ | c | >{\centering\arraybackslash}p{2cm}|>{\centering\arraybackslash}p{2cm}|>{\centering\arraybackslash}p{2cm}|}
    \hline
      & Typ pouzdra & Ks v balení & Cena [Kč] \\ \hline
    LM13700M/NOPB & SOIC & 48 & 32.50\\ \hline
    LM13700MX/NOPB & SOIC & 2500 & 27.04\\ \hline
    LM13700N/NOPB & PDIP & 25 & 48.62\\ \hline
  \end{tabular}
  \caption[Porovnání typů LM13700]{\label{tab:Porovnání typů LM13700}Porovnání typů LM13700 \cite{18}}
  \end{table}
\begin{figure}[h]
\centering
\includegraphics[scale=0.55]{image6.png}
\caption[Pinout LM13700M]{Pinout LM13700M \cite{20} \label{s:PIN}}
\end{figure}
\noindent Vnitřní zapojení LM13700 na obrázku \ref{s:OTA} obsahuje symetrický rozdílový zesilovací stupeň (tranzistory Q4, Q5), který je napájen řízeným zdrojem proudu s tranzistorem Q2. Tento diferenční stupeň pracuje jako měnič vstupního rozdílového napětí na diferenční proudový signál, který je převeden proudovými zrcadly (\textit{Current Mirror}) na výstupní svorky obvodu. Proudová zrcadla zde tvoří dvojice diod a tranzistorů --- referenční proud tekoucí v jedné větvi obvodu se \uv{zrcadlí} ~v jeho druhé větvi. Principiálně jsou to zdroje proudu řízené proudem (viz Motchenbacher, Connelly \cite{21}).
\\
\\noindent Významnou vlastností OTA je možnost změny transkonduktance $g_m$ změnou klidového stejnosměrného pracovního proudu vstupního diferenčního stupně. Řízení může být buď napěťové nebo proudové.
\begin{figure}[h]
\centering
\includegraphics[scale=0.75]{image5.png}
\caption[Vnitřní schéma OTA]{Vnitřní schéma OTA \cite{20}\label{sec:OTA}}
\end{figure}
\sekce{Náhrada prvků}\label{s:NAH}
\noindent Pro ideální OTA zesilovač (vstupní i výstupní impedance nulové) je možno odpor nahradit obvodem s uzemněným neinvertujícím vstupem a zpětnou vazbou z invertujícího vstupu na výstup a to hodnotou
\begin{align}
R_{in} = \frac{1}{g_{m}},
\end{align}
kde $g_{m}$ označuje transkonduktanci zesilovače. Prohození invertujícího a neinvertujícího vstupu vede na opačnou polaritu. Tato konfigurace jako odpor je užitečná např. k návrhu monolitických $g_m$-RC filtrů pouze s transkonduktancemi a kapacitami ($g_m$-C filtry - viz \ref{s:OTA}, \ref{s:GM-C}), také k náhradě velmi velkých odporů.
\begin{figure}[h]
\centering
\includegraphics[scale=0.7]{image10.png}
\caption[Náhradní obvod pro uzemněný rezistor]{Náhradní obvod pro uzemněný rezistor \cite{12}}
\end{figure}
\noindent Pro nahrazení indukčnosti o impedanci $Z_L = 1/(sC)$ lze použít obvod s třemi OTA. Uzemněny jsou invertující vstup prvního OTA a neinvertující druhého. Použita je zpětná vazba z výstupu na neinvertující vstup prvního OTA. Propojení výstupu prvního OTA na invertující vstup druhého OTA je realizován přes uzemněný kapacitor. \\
Vyjádřením napětí a proudů v obvodu bylo získáno napětí na kapacitoru a vstupní proud
\begin{align}\label{s:vzt2}
V_C &= \frac{g_{m1}}{sC}V_1 \\
I_1 &= g_{m2}V_C = \frac{g_{m1}g_{m2}}{sC}V_1.
\end{align}
Výsledná indukčnost - impedance vstupu byla vyjádřena vztahem \ref{s:vzt2}.
\begin{align}
Z_{in}(s) = \frac{V_1}{I_1} = s\frac{C}{g_{m1}g_{m2}}
\end{align}
\noindent Byl obdržen induktor o hodnotě
\begin{align}
L = \frac{C}{g_{m1}g_{m2}}.
\end{align}
\begin{figure}[h]
\centering
\includegraphics[scale=0.55]{image13.png}
\caption[Náhradní obvod pro indukčnost]{Náhradní obvod pro indukčnost \cite{12} \label{s:IND}}
\end{figure}
\noindent Pro uzemněnou indukčnosti o impedanci $Z_L = 1/(sC)$ byl použit obvod na Obrázku \ref{s:IND}. Vyjádřením napětí a proudů v obvodu bylo získáno napětí na kapacitoru a vstupní proud
\begin{align}
V_C &= \frac{g_{m1}}{sC}V_1 \\
I_1 &= g_{m2}V_C = \frac{g_{m1}g_{m2}}{sC}V_1.
\end{align}
Výsledná indukčnost - impedance vstupu byla vyjádřena vztahem
\begin{align}
Z_{in}(s) = \frac{V_1}{I_1} = s\frac{C}{g_{m1}g_{m2}}.
\end{align}
\begin{figure}[h]
\centering
\includegraphics[scale=1]{image12.png}
\caption[Náhradní obvod pro neuzemněnou indukčnost]{Náhradní obvod pro neuzemněnou indukčnost pro $g_{m1} = g_{m2}$\cite{12}}
\end{figure}
\subsection{Základní bloky}
\begin{figure}[h]
\centering
\includegraphics[scale=0.55]{otos.png}
\caption[Základní bloky s OTA]{Základní bloky s OTA a) integrující b) srovnávací c) sčítací \cite{12} \label{s:BLO}}
\end{figure}
\noindent Blok a) na obrázku \ref{s:BLO} slouží k realizaci invertujícího/neinvertujícího integrátoru s výsledným napětím
\begin{equation}
V_O = \frac{g_{m1}}{pC}(V_1 - V_2).
\end{equation}
Blok b) na obrázku \ref{s:BLO} je komparátor s různou polaritou a napětím na výstupu
\begin{equation}
V_O = \frac{g_{m1}}{g_{m2}}(V_1 - V_2).
\end{equation}
Blok c) na obrázku \ref{s:BLO} realizuje sčítací nebo rozdílový obvod s napětím na výstupu
\begin{equation}
V_O = -\frac{g_{m1}}{g_{m3}}V_1 + \frac{g_{m2}}{g_{m3}}V_2.
\end{equation}
\noindent Spojením těchto základních stavebních bloků se správnými znaménky lze získat různé funkční bloky.\\
Základním principem uplatňovaným při návrhu s OTA je použití pouze OTA a uzemněných kapacitorů, protože při návrhu IC (\textit{Integrated Circuit}) jsou uzemněné kapacitory méně zatíženy parazitními chybami než neuzemněné kapacitory. Pro IC použití je vhodné volit shodné transkonduktance. Parazitní vstupní a především výstupní impedance způsobují chyby ve výstupu filtru, což může vést na parazitní póly, které při vysokofrekvenčním použití nelze zanedbat. Při použití filtru pro zvukové aplikace (20 -- 20 000 Hz) lze chyby způsobené parazitními součástkami zanedbat, rovněž lze zanedbat chyby způsobené konečnou šířkou pásma.
\subsection{Odvození DP 2. řádu}\label{s:ODV}
\noindent Náhradní obvod, ze kterého bude spočítána přenosová funkce pro přenos filtru druhého řádu, popisuje Obrázek \ref{s:DPO}.
\begin{figure}[h]
\centering
\includegraphics[scale=0.15]{RLC_low-pass.png}
\caption[Dolní propust 2. řádu]{Dolní propust 2. řádu \cite{11} \label{s:DPO}}
\end{figure}
\noindent Přenos obvodu byl vyjádřen jako
\begin{align}
H(s) = \frac{U_{out}}{U_{in}} = \frac{Z_2}{Z_1}, \quad Z_1 = sL,\quad Z_2 = \frac{\frac{R}{sC}}{R + \frac{1}{sC}}.
\end{align}
Výsledný přenos je roven 
\begin{align}
H(s) = \frac{\frac{\frac{R}{sC}}{R + \frac{1}{sC}}}{sL + \frac{\frac{R}{sC}}{R + \frac{1}{sC}}}.
\end{align}
Elementárními algebraickými úpravami a následným vynásobením členem $1/(LRC)$ byl získán výsledný přenos.
\begin{align}\label{s:vzt4}
H(s) = \frac{R}{s^2LRC + sL + R} = \frac{\frac{1}{LC}}{s^2 + \frac{s}{RC} + \frac{1}{LC}}.
\end{align}
\noindent Využitím poznatků ze Sekce \ref{s:NAH} je možno za odpor a indukčnost dosadit do vztahu \ref{s:vzt4}. Byly uvažovány kapacitory o stejné hodnotě C.
\begin{align}
H(s) = \frac{\frac{1}{\frac{C^2}{g_{m1}g_{m2}}}}{s^2 + \frac{s}{\frac{C}{g_{m2}}} + \frac{1}{\frac{C^2}{g_{m1}g_{m2}}}} = \frac{\frac{g_{m1}g_{m2}}{C^2}}{s^2 + \frac{sg_{m2}}{C} + \frac{g_{m1}g_{m2}}{C^2}} = \frac{g_{m1}g_{m2}}{s^2C^2 + sg_{m2}C + g_{m1}g_{m2}}.
\end{align}
Porovnáním jmenovatele se jmenovatelem přenosu filtru 2. řádu byl obdržen vztah
\begin{align}
s^2 + s\frac{\omega _c}{Q} + \omega _c^2 &= s^2C^2 + sg_{m2}C + g_{m1}g_{m2}\\
s^2 + s\frac{\omega _c}{Q} + \omega _c^2 &= s^2 + \frac{sg_{m2}}{C} + \frac{g_{m1}g_{m2}}{C^2}.
\end{align}
Z tohoto vztahu byl vyjádřen mezní kmitočet jako 
\begin{align}
\omega _c^2 &= \frac{g_{m1}g_{m2}}{C^2} \\
\omega _c &= \sqrt{\frac{g_{m1}g_{m2}}{C^2}}
\end{align}
a činitel jakosti dosazením za $\omega _c$
\begin{align}
Q = \frac{\omega _c}{\frac{g_{m2}}{C}} = \sqrt{\frac{g_{m1}}{g_{m2}}}.
\end{align}
Pokud navíc byly uvažovány stejné transkonduktance $g_{m1}, \ g_{m2} = g_m$, byl obdržen výsledek
\begin{align}
\omega _c &= \sqrt{\frac{g_m^2}{C^2}},\\
Q &= \sqrt{1} = 1.
\end{align}
\sekce{Simulace}
\subsection{Dolní propust 2. řádu}\label{s:DP2}
Dolní propust druhého řádu má přenos v nekonečnu nulový $H_{\infty} = 0$. Přenosová funkce je
\begin{align}
H(j\omega) = \frac{H_0 \omega_c ^2}{(j\omega)^2 + \frac{\omega _c}{Q}(j\omega) + \omega _c ^2}.
\end{align}
\noindent Obvodová simulace byla realizována v programu Multisim. Bylo zvoleno symetrické napájení OZ $V_{DD},V_{SS} = \pm 15$ V. Regulací vstupního proudu je ovlivňován pracovní bod obvodu (mezní kmitočet). Vstupní externí proud $I_{ABC} = 0.5$ $\mu$A byl zvolen tak, aby byl obdržen mezní kmitočet cca 100 kHz. Externím proudem $I_{ABC} \in$ $[5$ $\mu$A ; 500 $\mu$A] je garantováno minimální výstupní napětí $U_{OUT} = \pm 12$ V, standardně $V_{peak 1} = 14.2$ V a $V_{peak 2} = -14.4$ V. Při výstupním napětí v tomto intervalu je šum vzhledem k signálu zanedbatelný a nezkreslí výsledky simulace.\\
\noindent Bylo použito zapojení s paralelně řazeným uzemněným kapacitorem a odporem a indukčností. Na výstupu 2. OTA (V1) byl obdržen filtr typu PP 1. řádu. Na výstupu 3. OTA (V2) pak DP 2. řádu.
\begin{figure}[H]
\centering
\includegraphics[scale=0.3]{bplp.png}
\caption{Schéma zapojení DP 2. řádu}
\end{figure}\begin{figure}[H]
\centering
\includegraphics[scale=0.5]{bplp2.png}
\caption{Amplitudová a fázová charakteristika DP 2. řádu, PP}
\end{figure}
\subsection{Dolní propust 4. řádu}\label{s:DP4}
Kaskádní zapojení sestává ze sériově zapojených bloků.
\begin{figure}[H]
\centering
\includegraphics[scale=0.4]{schemata.png}
\caption{Kaskádní zapojení \cite{5}}
\end{figure}
\noindent Přenosové funkce jednotlivých bloků se násobí
\begin{align}
H_k(j\omega) = \frac{U_k (j\omega)}{U_{k-1}(j\omega)}.
\end{align}
Přenos posledního bloku je dán vztahem
\begin{align}
H_{1 \rightarrow k}(j\omega) = \frac{U_k (j\omega)}{U_{in}(j\omega)} = \prod _{n=1}^{k} H_n(j\omega).
\end{align}
Kaskádním zapojením dvou dolních propusti ze sekce \ref{s:DP2} byl obdržen filtr 4. řádu s poklesem -80 dB/dek.
\begin{figure}[H]
\centering
\includegraphics[scale=0.5]{lpbp32.png}
\caption{Schéma zapojení DP 4. řádu}
\end{figure}\begin{figure}[H]
\centering
\includegraphics[scale=0.45]{bplp3.png}
\caption{Amplitudová a fázová charakteristika DP 4. řádu}
\end{figure}
\subsection{Pásmová propust 2. řádu}\label{s:PP2}
Horní propust druhého řádu má přenos v nule nulový $H_{0} = 0$. Přenosová funkce je
\begin{align}
H(j\omega) = \frac{H_{\infty} (j\omega) ^2}{(j\omega)^2 + \frac{\omega _c}{Q}(j\omega) + \omega _c ^2}.
\end{align}
\noindent Pásmovou propust lze získat zapojením dolní a horní propusti.
\begin{figure}[H]
\centering
\includegraphics[scale=0.9]{fig9.png}
\caption{Násobení přenosů \cite{13}}
\end{figure}
\noindent Pásmová propust má přenos v nule i nekonečnu nulový $H_{0} = H_{\infty} = 0$. Přenosová funkce je
\begin{align}
H(j\omega) = \frac{H_{B} \frac{\omega _c}{Q} (j\omega) }{(j\omega)^2 + \frac{\omega _c}{Q}(j\omega) + \omega _c ^2}.
\end{align}
\begin{figure}[H]
\centering
\includegraphics[scale=0.6]{PP2O.png}
\caption{Schéma zapojení PP 2. řádu}
\end{figure}
\begin{figure}[H]
\centering
\includegraphics[scale=0.6]{PP2O2.png}
\caption{Amplitudová a fázová charakteristika PP 2. řádu}
\end{figure}
\subsection{Pásmová propust 4. řádu}\label{s:PP4}
\noindent Zapojením dvou PP 2. řádu byla obdržena PP 4.řádu s poklesem -80 dB/dek. 
\begin{figure}[H]
\centering
\includegraphics[scale=0.5]{PP4O.png}
\caption{Schéma zapojení PP 4. řádu}
\end{figure}
\begin{figure}[H]
\centering
\includegraphics[scale=0.6]{PP4O2.png}
\caption{Amplitudová a fázová charakteristika PP 4. řádu}
\end{figure}
\sekce{Návrh v Maple}\label{s:MAPLE}
V celé sekci je zvolena přesnost na 3 desetinná místa. Přesnější hodnoty jsou k nahlédnutí v přiloženém skriptu. Pro zachování přehlednosti nejsou v práci všechny výstupy z Maplu.\\
Pro návrh pásmové propusti 4. řádu s Cauerovou aproximací typu C byly zvoleny parametry tolerančního schématu 
\MapleOutput{f\_s := 60000 Hz}
\MapleOutput{f\_p := 150000 Hz}
\MapleOutput{fp := 190000 Hz}
\MapleOutput{fs := 280000 Hz}
\MapleOutput{ap := 1 dB}
\MapleOutput{as := 80  dB,}
\noindent kde všechny parametry musí být kladná reálná čísla a f\_s <  f\_p < fp < fs a ap < as. Zadána byla spodní a horní hranice nepropustného pásma $f\_s,fs$ [Hz], spodní a horní hranice propustného pásma $f\_p,fp$ [Hz], maximální útlum v propustném pásmu $ap$ [dB] a minimální útlum v nepropustném pásmu $as$ [dB]. Toleranční schéma definuje oblasti, do kterých nesmí charakteristika filtru zasáhnout.
\begin{align}
f\_s &= \frac{\sqrt{\Delta{fs}^2+4f\_m ^2}-\Delta{fs}}{2}\\
f\_p &= \frac{\sqrt{\Delta{fp}^2+4f\_m ^2}-\Delta{fp}}{2}\\
fp &= \frac{\sqrt{\Delta{fp}^2+4f\_m ^2}+\Delta{fp}}{2}\\
fs &= \frac{\sqrt{\Delta{fs}^2+4f\_m ^2}+\Delta{fs}}{2}
\end{align}
Funkcí $BP2NLP$ byla provedena transformace tolerančního schematu nesymetrické pásmové propusti (PP) na toleranční schema normované dolní propusti (NDP). Byl spočítán nový kmitočet pro horní hranici nepropustného pásma $f_s$. Byl spočten geometrický střed propustného pásma $f_m$ [Hz], šířka propustného pásma $\Delta{fp}$ [Hz] a šířka nepropustného pásma $\Delta{fs}$ [Hz].
\MapleOutput{f\_s = 101785.714 Hz}
\MapleOutput{fm = 168819.43 Hz}
\MapleOutput{\Delta{fp} = 40000 Hz}
\MapleOutput{\Delta{fs} = 178214.286 Hz}
\noindent Byl obdržen kmitočet hranice nepropustného pásma normované dolní propusti (NDP) $Os$ [1/s].
\MapleOutput{Os = 4.455 1/s.}
\begin{figure}[h]
\centering
\includegraphics[scale=0.5]{tolsch2.png}
\caption{Toleranční schéma navrhované pásmové propusti}
\end{figure}
\noindent Stupeň Cauerovy aproximace normované dolní propusti byl určen jako $order = 4$. Pro sudý stupeň Cauerovy aproximace jsou definovány tři typy - A, B, C. Tyto typy se od sebe liší průběhem aproximační funkce. Byla zvolena aproximace typu C se shodnými zakončovacími odpory.
\noindent Dále byla funkcí $Cauer\_asnew$ určena nová hodnota útlumu v nepropustném pásmu NDP.
\MapleOutput{asnew := 81.719 dB}
\begin{align}
asnew&= 10 \cdot log_{10}\left(1 + \left( \frac{\epsilon}{kl\_new}\right)^2\right)\\
\epsilon &= \sqrt{10^{0.1ap} - 1}\\
k &= \frac{1}{Os}\\
kl\_new &= k^{order}\left(\prod_{i=1}^{n}JacobiCD\left(\frac{(2i - 1 + m)EllipticK(k)}{order},k\right)\right)^4,
\end{align}
\noindent kde $m$ je celočíselný zbytek po dělení řádu 2 a $n$ celočíselný výsledek dělení. Jakobiho eliptických funkcí je 12 a vycházejí ze škálování na jednotkové elipse (cos $\phi$, sin $\phi$ se neváží k jednotkovému kruhu, ale k elipse). \\
\\
Následně byl spočten koeficient nejvyšší mocniny polynomu ve jmenovateli přenosové funkce $Gc$, póly a nuly přenosové funkce $poles, zeros$ pomocí funkce $CauerCPolesZeros$. Počet pólů je dán řádem filtru $order$ a počet nul pro aproximaci typu C je roven $order = 2$. Dále byla spočtena Caurerova aproximace typu C - provozní činitel přenosu $G$ jako racionální lomená funkce $G(p) = 1/H(p)$, charakteristická funkce $chf$ jako $\Phi(p)$ s nulami a póly na imaginární ose a nuly přenosu. Charakteristická funcke má shodný jmenovatel s $G(p)$.
\MapleOutput{Gc, poles, zeros := 94.811,}
\MapleOutput{[-0.478 + 0.343 I, -0.478 - 0.343 I, -0.161 + 0.983 I, -0.161 - 0.983 I]),}
\MapleOutput{[5.706 I, -5.706 I]),}
\MapleOutput{G,chf,zer := \frac{94.881p^4+121.138p^3+156.142p^2+100.507p+32.556}{p^2+32.556}, \frac{(94.811p^2+78.754)p^2}{p^2+32.556}, [5.706 I, -5.706 I]}
\begin{figure}[h]
\centering
\includegraphics[scale=0.5]{sch02.png}
\caption{Modulová frekvenční charakteristika NDP}
\end{figure}
\noindent Charakteristika byla vykreslena z přenosu funkcí $MagnitudeHdB$, která vypočte modul přenosu podle předpisu |H(j$\omega)$| a výsledek převede na $20 \cdot log_{10} |H(p)|$.
\subsection{Příčkové LC filtry}\label{s:LC}
Pasivní dolní propust je realizována zapojením induktoru ke vstupnímu napětí a k této větvi je následně zapojen paralelně rezistor. Pasivní horní propust má ke vstupu připojený sériově rezistor a poté k této větvi paralelně induktor. \\
\\
K realizaci filtrů vyšších řádů se užívají $\pi$ nebo T~články s LC prvky. Podle Vedrala a Svatoše \cite{8} musí být při návrhu filtru zohledněn vnitřní odpor zdroje $R_s$ a zatěžovací odpor $R_L$. LC filtry jsou tedy dvojitě zakončeny. Indukčnosti a kapacity prvků se určí z rovnic pro normované kapacity a indukčnosti. Normované hodnoty budou vypočteny pro mezní kmitočet $\omega _0 = 1/\sqrt{LC}$ a pro zatěžovací odpor $R_L$. Hodnoty prvků lze pro požadovanou aproximaci odečíst z tabulek. Pro LC filtry se používá kmitočtová oblast $10^{3}$~Hz--$10^{2}$~MHz.\\
\\
\begin{figure}[h]
\centering
\includegraphics[scale=0.13]{piclanky.png}
\caption[Pasivní dolní propust n-tého řády s $\pi$ články]{Pasivní dolní propust n-tého řády s $\pi$ články \cite{8}}
\end{figure}
\begin{figure}[h]
\centering
\includegraphics[scale=0.08]{tclanky.png}
\caption[Pasivní dolní propust n-tého řády s T články]{Pasivní dolní propust n-tého řády s T články \cite{8}}
\end{figure}
\subsection{Gyrátory}\label{s:GYR}
\noindent K převodu induktoru na zapojení s kapacitorem byla použita struktura označovaná jako gyrátor. Jde o náhradu původního obvodu s induktorem vhodným uspořádáním rezistorů a kapacitorů tak, že výsledná impedance vypadá jako induktor. Jelikož po této substituci v obvodu zůstanou jen R,C prvky, jedná se o ARC syntézu. \\
\\
 Gyrátor je podle Martinka, Boreše a Hospodky \cite{12} typ invertoru. Pro invertory platí, že jejich vstupní impedanci lze napsat ve tvaru\begin{equation}
Z_{vst} = \frac{a_{12}}{a_{21}}\frac{1}{Z_L} = \frac{a_{12}}{a_{21}}Y_L.
\end{equation}
Pokud jsou parametry $a_{12}, a_{21}$ reálné a kladné, hovoříme o gyrátoru. Symbol gyrátoru je na obrázku \ref{s:G}. Gyrátor se nejúspěšněji dá realizovat paralelním spojením dvou napětím řízených zdrojů proudu s opačným znaménkem. Zapojení s OTA odpovídá dvěma zesilovačům, jeden s uzemněnou zápornou a druhý s uzemněnou kladnou svorkou vstupu. Výstup prvního ze zesilovačů je propojen s volnou vstupní svorkou druhého a naopak.
\begin{figure}[h]
\centering
\includegraphics[scale=0.1]{gyr.png}
\caption{Definice gyrátoru \label{s:G}}
\end{figure}
Podle Schaumanna a Valkenburga \cite{13} nelze gyrátor dobře realizovat s obyčejnými operačními zesilovači, běžně se používají \textit{General Impedance Converters} (GIC). Převod induktoru na jiné zapojení s ekvivalentní impedancí má praktické využití v integrovaných obvodech, kde jsou kapacitory preferovány nad induktory kvůli malým rozměrům. Navíc se induktory musí složitě vyrábět na danou hodnotu. V návrhu integrovaných obvodů se také většinou nepoužívají rezistory kvůli místu na čipu, které zabírají. \\
\\
Gyrátor je principielně spojení invertujícího a neinvertujícího napětím řízeného zdroje proudu, a proto ho lze realizovat snadno s transkonduktančními zesilovači. Na obrázku \ref{s:GO} jsou podle Schaumanna a Valkenburga \cite{13} znázorněny dva gyrátory s kapacitorem. 
\begin{figure}[h]
\centering
\includegraphics[scale=0.45]{gyrator.png}
\caption[Neuzemněný induktor realizovaný kapacitorem a dvěma gyrátory]{Neuzemněný induktor realizovaný kapacitorem a dvěma gyrátory \label{s:GO}}
\end{figure}
Obvodovou analýzou v uzlu V byla obdržena rovnice
\begin{align}
pCV &= g_mV_1 - g_mV_2
\end{align}
a dva proudy na výstupu
\begin{align}
I_1 = I_2 = g_mV.
\end{align}
Zkombinování rovnic a eliminace V vede k rovnici neuzemněného induktoru mezi napětími $V_1$ a $V_2$.
\begin{align}
I_1 = I_2 = \frac{g_m^2}{pC}(V_1 - V_2) = \frac{1}{pL}(V_1 - V_2)
\end{align}
Z rovnice lze snadno odvodit, že kapacita kondenzátoru použitého jako náhrada zapojení induktoru v zapojení s OTA je rovna $C_L = L g_m ^2$. \\
\subsection{Výpočet prvků LC filtru a přenosových funkcí}\label{s:VYP}
\noindent Funkcí $DroppNLP$ byly vypočteny prvky LC příčkového filtru typu normovaná dolní propust (NDP). Zakončení bylo zvoleno standardní (common), odpory o hodnotě 1 $\Omega$, směr zpracování od posledního prvku (rear), s T strukturou (začíná zepředu podélným induktorem). Standardní (common) zakončení je oboustranné ($R_1~\neq~0, R_z~\neq~\infty$). Výstupem funkce je LC struktura s orientací prvků ve větvi podélně (direct) nebo příčně (shunt).
\MapleOutput{block (1), [orientation = direct, elements = {L1 = 1.571}, Z = p L1]}
\MapleOutput{block (2), [orientation = shunt, elements = {C1 = 1.542}, Z = \frac{1}{pC1}]}
\MapleOutput{block (3), [orientation = direct, elements = {C1 = 0.02, L1 = 1.522}, Z = \frac{1}{\frac{1}{pL1} + pC1}]}
\MapleOutput{block (4), [orientation = shunt, elements = {C1 = 1.545}, Z = \frac{1}{pC1}]}
\noindent Přenosová funkce pasivních a aktivních struktur filtru byla spočtena funkcí $MakeH$. Byl spočten  napěťový i výkonový přenos. Z rozložení pólů je patrné, že obě přenosové funkce jsou stabilní.
\begin{align}
190.352s_1^4 + 242.742s_1^3 + 312.889s_1^2 + 201.21s_1 + 65.112 &= 0 \\
95.176s_2^4 + 121.371s_2^3 + 156.444s_2^2 + 100.605s_2 + 32.556 &= 0
\end{align}
\begin{align}
s_1 &= {-0.477 - 0.3431 I}, {-0.477 + 0.343 I}, {-0.161 - 0.983 I}, {-0.161 + 0.983 I}\\
s_2 &= {-0.477 - 0.3431 I}, {-0.477 + 0.343 I}, {-0.161 - 0.983 I}, {-0.161+ 0.983 I}
\end{align}
\MapleOutput{H\_NLPV := \frac{p^2  + 32.556}{190.352p^4  + 242.742p^3  + 312.889p^2  + 201.21p + 65.112}}
\MapleOutput{H\_NLP := \frac{p^2  + 32.556}{95.176p^4 + 121.371p^3 + 156.444p^2 + 100.605p + 32.556}}
\begin{figure}[h]
\centering
\includegraphics[scale=0.5]{sch022.png}
\caption{Modulová frekvenční charakteristika NDP - LC příčkový filtr}
\end{figure}
\noindent Hodnota přenosové funkce v 1 byla vyhodnocena jako $-1.007$.\\
Byla provedena transformace hodnot prvků normované dolní propusti (NDP) na pásmovou propust (PP). Zakončovací rezistor byl zvolen 1 $\Omega$, další dva parametry funkce značí spodní a horní hranici propustného pásma.
\MapleOutput{block (1), [Z = pL1 + \frac{1}{pC1}, orientation = direct, elements = {C1 = 1.422*10^{-7}, L1 = 6.252*10^{-6}}]}
\MapleOutput{block (2), [Z = \frac{1}{\frac{1}{pL1}+pC1}, orientation = shunt, elements = {C1 = 6.135*10^{-6}, L1 = 1.449*10^{-7}}]}
\MapleOutput{block (3), [Z = \frac{1}{pC1 + \frac{1}{pL1}+\frac{1}{pL2 + \frac{1}{pC2}}} orientation = direct, elements = { C1 = 8.031*10^{-8} ,}}
\MapleOutput{C2 = 1.468*10^{-7}, L1 = 1.107*10^{-5}, L2 = 6.055*10^{-6}]}
\MapleOutput{block (4), [Z = \frac{1}{\frac{1}{pL1}+pC1}, orientation = shunt, elements = { C1 = 6.149*10^{-6}, L1 = 1.445*10^{-7}}]}
\noindent Vygenerovaná struktura je popsána na obrázku \ref{s:SCHEM}.
\begin{figure}[h]
\centering
\figppp{Circuit(1)}{5.566}{2.075}{}{}
\caption{Schéma LC příčkové struktury \label{s:SCHEM}}
\end{figure}
\noindent Byly nastaveny jakosti cívek v LC příčkové struktuře na konečnou hodnotu. Funkce $MakeRealL$ zařadí do výsledné LC příčkové struktury sériově rezistory k induktorům podle zadaného činitele jakosti $Q$ a zadaného kmitočtu (ten odpovídá u pásmové propusti geometrickému středu propustného pásma - nebo je možno zadat obě hranice propustného pásma). Byl zvolen činitel jakosti 100. Činitel jakosti je dán převrácenou hodnotou poměrné šířky pásma
\begin{equation}
Q = \frac{1}{B} = \frac{\omega_s}{\delta \omega},
\end{equation}
kde B je poměrná šířka pásma, $\delta \omega = \omega_2 - \omega_1$ a $\omega_s = \sqrt{\omega_1\omega_2}$. $\omega_1$ a $\omega_2$ zde jsou mezní kruhové kmitočty odpovídající poklesu přenosu filtru o 3~dB.\\
\\
Pro kmitočtové pásmo stovky kHz až jednotky MHz v závislosti na typu jádra a kvalitě materiálu lze dosahovat hodnoty činitele jakosti cca 1000 a hodnoty indukčnosti řádově 100 $\mu$H až 10 mH. Pro činitel jakosti se zde uplatňuje kmitočtová závislost $Q = \omega L/R$. Pro kmitočtové pásmo do 10 kHz hodnoty činitele jakosti klesají řádově na hodnoty 10 pro velké hodnoty L. Výpočet sériového odporu je proveden podle předpisu $R_s~=~L1~\cdot~2~\pi~f/Q$.
\MapleOutput{block (1), [Z = pL1 + Rs1 + \frac{1}{pC1}, orientation = direct, elements = {Rs1 = 0.066}]}
\MapleOutput{block (2), [Z = \frac{1}{\frac{1}{pL1 + Rs1}+pC1}, orientation = shunt, elements = {Rs1 = 0.002}]}
\MapleOutput{block (3), [Z = \frac{1}{pC1 + \frac{1}{pL1 + Rs1}+\frac{1}{pL2 + Rs2 + \frac{1}{pC2}}} orientation = direct, elements = {Rs1 = 0.117, Rs2 = 0.064}]}
\MapleOutput{block (4), [Z = \frac{1}{\frac{1}{pL1 + Rs1}+pC1}, orientation = shunt, elements = {Rs1 = 0.002}]}
\noindent Byl spočten přenos pro LC strukturu bez a s přidanými sériovými rezistory. Pro oba přenosy byla vykreslena modulová frekvenční charakteristika. Přenosové funkce zde pro svou složitost a zachování přehlednosti textu nejsou uváděny, ale jsou k nalezení v přiloženém Maple skriptu.
\begin{figure}[h]
\centering
\includegraphics[scale=0.6]{modul12.png}
\caption{Modulová frekvenční charakteristika LC struktury (červená) a LC struktury s konečnou hodnotou jakostí cívek (zelená)}
\end{figure}
\begin{figure}[h]
\centering
\includegraphics[scale=0.6]{modul123.png}
\caption{Přiblížená modulová frekvenční charakteristika LC struktury (červená) a LC struktury s konečnou hodnotou jakostí cívek (zelená)}
\end{figure}
\noindent Vyčíslením v $f_m \cdot 2 \pi$ Hz, kde $f_m$ je geometrický střed propustného pásma, bylo obdrženo zesílení 0.283 dB.\\
\\
\subsection{Simulace prvků LC prototypu}\label{s:ARC}
V této sekci byl náhradou induktorů v LC prototypu za gyrátory obdržen návrh ARC filtru.\\
Zatím neodnormované prvky byly vyčísleny následovně
\MapleOutput{ele\_BP := { C1 = 1.422*10^{-7}  , C2 = 6.135*10^{-6},  C3 = 1.468*10^{-7} , C4 = 8.031*10^{-8},}}
\MapleOutput{ C5 = 6.149*10^{-6}, L1 = 6.252*10^{-6}, L2 = 1.449*10^{-7}, L3 = 6.055*10^{-6}, L4 = 1.107*10^{-5}, }
\MapleOutput{L5 = 1.445*10^{-7}, R1 = 1, Rz = 1.}
\noindent Odnormované hodnoty kapacit získané vydělením kmitočtem $fp \cdot 2 \pi$, kde $fp$ je horní hranice propustného pásma, byly spočteny jako
\MapleOutput{C1 = 1.191*10^{-13}, C2 = 5.139*10^{-12}, C3 = 1.23*10^{-13}, C4 = 6.727*10^{-14}, C5 = 5.151*10^{-12}.}
\noindent Frekvenčně a impedančně odnormované odpory byly vypočteny podělením kmitočtem $fp \cdot 2 \pi$ a přibližnou hodnotou kapacity pro mikroelektronickou realizaci $C = 2 pF$.
\MapleOutput{R1 = Rz = 418828.797}
\noindent Využitím poznatků ze sekce \ref{s:GYR} byly dosazením do vztahu $C = L \cdot g_m^2$ s uvažováním minimální transkonduktance z datasheetu LM13700 (gm = 9600 $\mu$S) získány kapacity 
\MapleOutput{CL1 = 5.762*10^{-10}, CL2 = 1.335*10^{-11}, CL3 = 5.58*10^{-10}, CL4 = 1.02*10^{-15}, CL5 = 1.332*10^{-11}.}
\noindent Výsledné hodnoty všech součástek s přesností na dvě desetinná místa jsou $C1 = 11.91$~pF, $C2 = 5.14$~pF, \\$C3 = 12.3$~pF, $C4 = 672.74$~pF, $C5 = 5.15$~pF, $C_{L1} = 57.62$~nF, $C_{L2} = 133.52$~nF, $C_{L3} = 55.8$~nF, \\$C_{L4} = 1019.9$~pF, $C_{L5} = 133.21$~nF, $R1 = Rz = 418.83\ k\Omega$.
\subsection{Funkční simulace LC protoypu}\label{s:KASK}
Základní myšlenka funkční simulace LC prototypu vychází z popisu příčkové struktury grafem signálových toků a simulací tohoto grafu vhodným elektronickým obvodem. Z aproximace bylo kaskádní syntézou získáno rozložení výsledné přenosové funkce filtru na funkce jednotlivých kaskádně řazených bloků. Kaskádně spojené dvojbrany se vzájemně neovlivňují - v napěťovém módu mají charakter napětím řízených zdrojů napětí a v proudovém módu proudem řízených zdrojů proudu. Přenos celé kaskády je dán součinem přenosů jednotlivých bloků. Na jeho základě se realizuje návrh filtru jako návrh jednotlivých bloků. Návrh je proveden s bloky s jedním OTA a po realizaci lze jednotlivé bloky modifikovat jak z hlediska struktury, tak z hlediska hodnot jednotlivých prvků vybrané struktury. Například změnou transkonduktance jednotlivých bloků pak lze variabilně modifikovat mezní kmitočet.\\
\\
\noindent Analýzou LC struktury z Maplu byly obdrženy obvodové rovnice, kde R je volitelný (fiktivní) rezistor
\begin{align}
I_1 &= \frac{1}{R_1 + pL_1 + \frac{1}{pC_1}}(U_G - U_2)\\
v_1 & = \frac{R}{R_1 + pL_1 + \frac{1}{pC_1}}(U_G - U_2)\\
U_2 &= \frac{1}{\frac{1}{pL_2} + pC_2}(I_1 - I_{3} - I_{L4} - pC_4 v_{L4})\\
U_2 &= \frac{1}{\frac{R}{pL_2} + RpC_2}(v_1 - v_{L3} - v_{L4} - RpC_4 U_{L4})\\
I_{3} &= \frac{1}{pL_3 + \frac{1}{pC_3}}(U_2 - U_3)\\
v_{L3} &= \frac{R}{pL_3 + \frac{1}{pC_3}}(U_2 - U_3)\\
v_{L4} &= \frac{1}{\frac{1}{pL_4}+pC_4}(I_1 - I_{L2} - pC_2U_2 - I_{3} - pC_4 (U_2 - U_3))\\
v_{L4} &= \frac{1}{\frac{R}{pL_4}+RpC_4}(v_1 - v_{L2} - RpC_2U_2 - v_{L3} - RpC_4 (U_2 - U_3))\\
U_3 &= \frac{1}{\frac{1}{R_z}+pC_5 + \frac{1}{pL_5}}(I_1 - I_{L2} - pC_2U_2)\\
U_3 &= \frac{1}{\frac{R}{R_z}+RpC_5 + \frac{R}{pL_5}}(v_1 - U_2 - RpC_2 U_2).
\end{align}
\noindent To odpovídá realizační struktuře s pěti bloky o přenosech $H_1, \ldots,H_5$
\begin{align}
H_1 & = \frac{R}{R_1 + pL_1 + \frac{1}{pC_1}},\\
H_2 &= \frac{1}{\frac{R}{pL_2} + RpC_2},\\
H_3 &= \frac{R}{pL_3 + \frac{1}{pC_3}},\\
H_4 &= \frac{1}{\frac{R}{pL_4}+RpC_4},\\
H_5 &= \frac{1}{\frac{R}{R_z}+RpC_5 + \frac{R}{pL_5}}.
\end{align}
\noindent Uvedené přenosy budou použity v analýze Pracanem.
\subsection{Simulace obvodu}
\noindent  Zapojení s OTA vychází z již uvedených principů v sekci \ref{s:NAH}. K simulaci byly použity vypočtené hodnoty ze sekce \ref{s:ARC}. \\
\\
\noindent Bylo použito zapojení se vstupním odporem $R_0$ řazeným paralelně ke zdroji (vhodnější pro funkční simulaci - Schaumann, Valkenburg \cite{13} str. 639) a nahrazení odporů bloky OTA. Výsledné napětí bylo odebíráno z uzlu $V_{22}$ označeného na obrázku \ref{s:V1}. Šířka propustného pásma byla pro klidový stejnosměrný pracovní proud $I_{ABC} = 50$ $\mu$A odečtena jako 110.75~kHz. Geometrický střed propustného pásma odpovídá 100~kHz. Přeladěním filtru změnou klidového stejnosměrného pracovního proudu na $I_{ABC} = 100$ $\mu$A byla obdržena šířka pásma 225.88~kHz a geometrický střed propustného pásma 200~kHz. \\
\\
Pro obdržení geometrického středu propustného pásma $f_m = 168.819$~kHz je třeba zvolit klidový stejnosměrný pracovní proud $I_{ABC} = 84.4$ $\mu$A. Byla odečtena šířka pásma 202.78~kHz. 
\begin{figure}[h]
\centering
\includegraphics[scale=0.425]{maple.png}
\caption{Výsledné schéma\label{s:V1}}
\end{figure}
\begin{figure}[h]
\centering
\includegraphics[scale=0.6]{output1.png}
\caption{Amplitudová a fázová charakteristika PP 4. řádu s $I_{ABC} = 50$ $\mu$A}
\end{figure}
\begin{figure}[h]
\centering
\includegraphics[scale=0.55]{Capture.png}
\caption{Amplitudová a fázová charakteristika PP 4. řádu s $I_{ABC} = 100$ $\mu$A}
\end{figure}
\begin{figure}[h]
\centering
\includegraphics[scale=0.55]{capture2.png}
\caption{Amplitudová a fázová charakteristika PP 4. řádu s $I_{ABC} = 84.4$ $\mu$A}
\end{figure}
\subsection{Vliv zátěže na funkci obvodu}
\noindent V simulaci se samozřejmě předpokládá, že všechny OTA zesilovače jsou ideální. Chování filtru ve výsledku ovlivní nedokonalosti reálných OTA (ztráty, parazitní chyby). Další nevýhodou jsou kondenzátory a jejich odchylka od jmenovité hodnoty - oproti tomu rezistory mají obecně minimální odchylku od jmenovité hodnoty. Z literatury podle Schaumanna a Valkenburga \cite{13} také víme, že reálné transkonduktance nejsou idální zdroje proudu (s nulovou výstupní admitancí) a že většina $gm$-C bloků použitých v obvodu má nenulové výstupní admitance. Jejich chování bude tedy extrémně závislé na zátěži, což může úplně změnit zamýšlenou funkci obvodu. Například ve sčítacím obvodu na obrázku \ref{s:BLO}, popsaným rovnicí \ref{s:BLO3}, zátěžová admitance $Y_L$ změní $g_{m3}$ na $g_{m3} + Y_L$. Podobně zátěž $Y_L$ na ztrátovém integrátoru na obrázku \ref{s:OTA-INT} (popsaný rovnicí \ref{s:OTA-INT1}) způsobí změnu $g_m$ na $g_m + Y_L$ v přenosové funkci integrátoru. Proto by transkonduktanční obvody obecně měly být navrženy tak, aby základní bloky řídily vysoko-impedanční uzly (např. vstupy jiných OTA). Pokud mají být řízeny velké zátěže, obvod s OTA musí být řízen \textit{bufferem} (v~pinoutu LM13700 na obrázku \ref{s:PIN} pin 7, 8 pro první OTA zesilovač a 9,10 pro druhý OTA zesilovač). Případně lze jako \textit{buffer} použít operační zesilovač s jednotkovým zesílením.\\
\\
\noindent K určení chování obvodu musíme mít podle Schaumanna a Valkenburga \cite{13} na paměti, že parazitní admitance $y_p = y_i + y_o$ je přítomna na každém uzlu spojujícím dva OTA zesilovače. Pokud pro jednoduchost předpokládáme, že všechny OTA jsou stejné a výstup $V_{out}$ je zatížen $Y_L$, dostaneme vztah
\begin{equation}
V_{out} = \frac{g_{m1}}{y_p}\frac{g_{m2}}{y_p}\frac{g_{m3}}{y_p} \ldots \frac{g_{mn}}{Y_L + y_p}(V_{in} - V_{out}).
\end{equation}
\noindent Po úpravě
\begin{align}
\frac{V_{out}}{V_{in}} &= \frac{1}{1 + (\frac{y_p}{g_m})^n(1 + \frac{Y_L}{y_p})} \simeq 1,\\
\lvert \frac{y_p}{g_m} \rvert ^n &\ll 1.
\end{align}
\noindent Podobně pro výstupní impedanci $Z_{out}(p)$ platí
\begin{align}
Z_{out}(p) &= \frac{\frac{1}{y_p}}{1 + (\frac{g_m}{y_p})^n} \simeq \frac{1}{y_p}(\frac{y_p}{g_m})^n,\\
\lvert \frac{y_p}{g_m} \rvert ^n &\ll 1.
\end{align}
\noindent Navrhnout transkonduktance tak, aby platilo $g_m \gg \lvert y_p \rvert$, $\lvert V_{out}/V_{in} \rvert  \simeq 1$ a $\lvert Z_{out} \rvert \simeq \lvert 1/y_p \rvert$ pro dostatečně velká n, je poměrně snadné. Obvykle se volí $n = 2$ nebo $n = 3$.
\subsection{Ladění filtru}
\noindent Pokud se analogový filtr má chovat podle specifikací, musí být navržen s přesnými hodnotami komponent. Podle Schaumanna a Valkenburga \cite{13} přenosová funkce závisí na frekvencích nul a pólů, Q faktoru pólů (Q faktor definuje, jak moc je systém podtlumený), zesílení - tyto parametry zase závisí na přesné hodnotě součástek. Kritické frekvence s jednotkami 1/čas jsou určeny absolutními hodnotami kapacitorů a rezistorů. Zesílení a Q faktor je určen poměrem kapacitorů a rezistorů. V diskrétních obvodech můžou být problémy vyřešeny laděním - buď před, nebo po dokončení návrhu. Pokud například máme časovou konstantu $T = RC$, můžeme změřit T a přizpůsobit rezistor (trimmerem), dokud neobdržíme požadovanou časovou konstantu $T_0$.\\
\\
Hlavním problémem ladění je přesné nalezení časové konstanty $C_U/g_m$, která mění mezní kmitočet. Časovou konstanta může být obdržena změnou $g_m$. Pokud je zesílení integrátoru jednotkové, časová konstanta bude nastavena na $1/\omega _{ref}$. Pokud bude referenční signál poslán na vstup integrátoru a oba vstupní a výstupní signály přes dva indentické špičkové detektory, naladíme $g_m$ dokud jednotkové zesílení frekvence integrátoru nebude $f_{ref}$. Tomuto zapojení se říká \textit{Master-Slave Tuning}. Také lze použít ladění pomocí Q-faktoru. \\
\\
Dalšími problémy obecně u OTA, které mohou ovlivnit funkci obvodu, je nízké stejnosměrné zesílení, nízké UGBW a vysoký šum. Tyto problémy se dají částečně vyřešit zvýšením transkonduktance. Zvýšením výstupní impedance se zvýší i stejnosměrné zesílení.
\subsection{Zhodnocení funkčnosti}
\noindent Pro návrh filtru ve spojitém časovém  pásmu se pro zhodnocení funkčnosti používá THD (\textit{Total Harmonic Distortion}) a SNR (\textit{Signal-to-Noise Ratio}). SNR lze definovat jako poměr mezi přijatým signálem a šumem spojeným se získáním tohoto signálu. Hodnota SNR roste s rostoucí velikostí signálu. Maximální hodnoty SNR je dosaženo při zaznamenání maximálního signálu (tzn. při dosažení úrovně saturace) \cite{21}. Poměr větší než 1 (0 dB) znamená, že amplituda signálu je větší než šumu. Definice je podle \cite{22}
\begin{equation}
SNR_{dB} = 10 \cdot log_{10}\left(\frac{P_{signal}}{P_{noise}}\right).
\end{equation}
 Hlavní vliv na THD má linearita transkonduktance, protože do systému filtru indukuje harmonické zkreslení (HD - \textit{Harmonic Distortion}). Pro nízké frekvence má na SNR vliv tepelný šum - ten může vzniknout vlivem nerovnoměrností struktury, teplotními kmity krystalové mřížky náhodným, či tepelným pohybem nabitých částic (zpravidla elektronů) v rámci elektrických vodičů. Teoreticky se dá říci, že tepelný šum není generován jen ve vodičích, jejichž teplota je rovna nebo téměř rovná absolutní nule. Jakákoliv vyšší teplota již znamená náhodný pohyb elektronů a tedy vznik šumu (Motchenbacher \cite{23}, \cite{24}). Vliv tepelného šumu ovlivňuje hlavně funkčnost OTA s menšími hodnotami transkonduktance. Tepelný šum je také znám jako 1/f šum, protože jeho spektrální hustota výkonu je inverzní k frekvenci \cite{1}. Ztráty způsobené šumem mohou být vzhledem ke konečnému zesílení kompenzovány předzesilovačem. \\
\\
Některá zapojení s nekonečnou vstupní imepedancí mají poměrně vysokou výstupní impedanci. Kaskádní zapojení lze rovněž kompenzovat \textit{bufferem}, což ale sníží šířku pásma celé struktury.\\
\\
Analýzou obvodu z obrázku \ref{s:PP4} a obvodu s hodnotami komponent z Maplu bylo obdrženo bylo určeno THD a šum. Byl použit klidový stejnosměrný pracovní proud $I_{ABC} = 50$ $\mu$A odpovídající geometrickému středu propustného pásma 100~kHz. Šum zde byl počítán jako výkon signálu ve zvoleném uzlu vydělený celkovým výkonem tepelného šumu na standardní teplotě (27 $^{\circ}$C). Je to tedy poměr vstupního SNR k výstupnímu SNR. Jak lze vidět z tabulek \ref{s:THD1} a \ref{s:THD2}, odstup signál šum je nejmenší pro frekvenci 100~kHz odpovídající geometrickému středu propustného pásma.
\begin{table}[h]
\centering
  \begin{tabular}{ | c | c | c |}
    \hline
     Frekvence [kHz] & Odečtený šum [dB] \\ \hline
    1 & 89.571 \\ \hline
    10 & 49.313 \\ \hline
    100 & 9.514 \\ \hline
    1000 & 45.341 \\ \hline
  \end{tabular}
  \caption[Šum pro PP 4. řádu (Maple)]{Šum pro PP 4. řádu (Maple) \label{s:THD1}}
\end{table}
  \begin{table}[h]
\centering
  \begin{tabular}{ | c | c | c |}
    \hline
     Frekvence [kHz] & Odečtený šum [dB] \\ \hline
    1 & 145.616 \\ \hline
    10 & 68.576 \\ \hline
    100 & 21.531 \\ \hline
    1000 & 108.632 \\ \hline
  \end{tabular}
\caption[Šum pro PP 4. řádu]{Šum pro PP 4. řádu (zapojení \ref{s:PP4}) \label{s:THD2}}
\end{table}
\noindent Také bylo změřeno THD pro různé frekvence zdroje se základní frekvenci 100~kHz, viz tabulka \ref{s:THD3} a \ref{s:THD4}. Byly zvoleny různé frekvence zdroje (1 a 100~kHz). Dle očekávání je nejnižší pro 100~kHz a co nejnižší zvolený počet harmonických frekvencí.
\begin{table}[h]
\centering
\renewcommand{\arraystretch}{1.15}
  \begin{tabular}{ | c | c | c |}
    \hline
    Frekvence zdroje [kHz] & Počet harmonických frekvencí & THD [\%] \\ \hline
    \multirow{3}{*}{1} & 3 & 60.115 \\& 5 & 68.136 \\& 10 & 74.169 \\ \hline
	\multirow{3}{*}{100} & 3 & 2.969\\& 5 & 3.102 \\& 10 & 3.125 \\ \hline
  \end{tabular}
  \caption[THD pro PP 4. řádu (Maple)]{THD pro PP 4. řádu (Maple) \label{s:THD3}}
\end{table}
\begin{table}[h]
\centering
\renewcommand{\arraystretch}{1.15}
  \begin{tabular}{ | c | c | c |} 
    \hline
     Frekvence zdroje [kHz] & Počet harmonických frekvencí & THD [\%] \\ \hline
    \multirow{3}{*}{1} & 3 & 55.679 \\& 5 & 62.926 \\& 10 & 68.599 \\ \hline
	\multirow{3}{*}{100} & 3 & 0.114\\& 5 & 0.216 \\& 10 & 0.267 \\ \hline
  \end{tabular}
  \caption[THD pro PP 4. řádu]{THD pro PP 4. řádu (zapojení \ref{s:PP4}) \label{s:THD4}}
\end{table}
\begin{figure}[h]
\centering
\includegraphics[scale=0.6]{thd.png}
\caption[THD analýza pro PP 4. řádu (Maple)]{THD analýza pro PP 4. řádu (Maple)}
\end{figure}
\begin{figure}[h]
\centering
\includegraphics[scale=0.6]{thd2.png}
\caption[THD analýza pro PP 4. řádu (zapojení \ref{s:PP4})]{THD analýza pro PP 4. řádu (zapojení \ref{s:PP4})}
\end{figure}
\sekce{Závěr}
\noindent Cílem práce bylo navrhnout pásmovou propust. Po~seznámení s principy OTA \ref{s:OTA} a náhradou prvků v obvodech s nimi \ref{s:NAH} byla provedena simulace. V sekci \ref{s:DP2} bylo v MultiSimu realizováno zapojení filtru typu dolní propust 2. řádu s poklesem 40\,dB/dek, pásmová propust 1. řádu s poklesem 20\,dB/\mbox{dek}. Poté byl kaskádním zapojením obdržen filtr typu dolní propust 4. řádu s poklesem 80\,dB/dek, pásmová propust 2. řádu s poklesem 40\,dB/dek. Dalším kaskádním blokem byla obdržena požadovaná pásmová propust 4. řádu s poklesem 80\,dB/dek. Výsledky simulací prokazují poměrně dobré vlastnosti navržené struktury.\\
\\
V sekci \ref{s:MAPLE} byla knihovnou Syntfil provedena matematická syntéza filtru a zapojení bylo převedeno na LC příčkovou strukturu. Byl proveden výpočet pro filtr 4. řádu s Butterworthovou aproximací. Mezní kmitočet a parametry propustného a zádržného pásma byly zvoleny $f_{-s} = 100$\,kHz, $f_{-p} = 140$\,kHz, 
$f_p$ = 160\,kHz, $f_s$ = 200\,kHz, kde $f_{-s}, f_s$ označuje spodní a horní hranici nepropustného pásma a $f_{-p}, f_p$ spodní a horní hranici propustného pásma. Útlum v propustném pásmu byl zvolen 1\,dB a v zádržném 40\,dB. Mezikrokem v návrhu byl převod pásmové propusti na normovanou dolní propust. Pro LC strukturu byly obdrženy hodnoty prvků, které byly odnormovány v sekci \ref{s:ARC123}. Byla provedena ARC syntéza s využitím gyrátorů, po níž výsledný obvod obsahoval pouze OTA a kapacitory. Výsledným zapojením vycházejícím z LC příčkového filtru bylo získáno 8 kapacitorů a 12 OTA (6 bloků LM13700). Byla provedena THD analýza a porovnání šumu pro různé frekvence. Z této analýzy se potvrdily dobré propustné vlastnosti filtru na kmitočtu 100\,kHz. Tento kmitočet byl pro účely simulace přeladěn klidovým stejnosměrným pracovním proudem z původních 150\,kHz. \\
\\
Knihovnou Syntfil byl spočítán i filtr 8. řádu s Cauerovou aproximací. Mezní kmitočet a parametry propustného a zádržného pásma byly zvoleny $f_{-s} = 60$\,kHz, $f_{-p} = 150$\,kHz, 
$f_p$ = 190\,kHz, $f_s$ = 280\,kHz. Útlum v propustném pásmu byl zvolen 1\,dB a v zádržném 80\,dB. Výsledným zapojením vycházejícím z LC příčkového filtru bylo získáno 10 kapacitorů a 15 OTA. Byla provedena THD analýza a porovnání šumu pro různé frekvence.\\
\\
Je nutné dbát na to, že toto zapojení obsahuje plovoucí kapacity a nebude vhodné pro krátké vlny (frekvence v řádech MHz, což odpovídá vlnovým délkám 10--100\,m). Pro tyto vysoké frekvence také OTA nemohou být použity kvůli limitovanému GBP. U plovoucích kapacit je také nutné dbát na to, že klidový stejnosměrný pracovní proud může způsobit akumulaci náboje na kapacitorech a eventuálně i saturaci OTA. Větší počet OTA také kvůli zpětným vazbám může mít vliv na stabilitu celého obvodu, čímž se sníží pásmo pro klidový stejnosměrný pracovní proud --- filtr pak může být stabilní jen v malém kmitočtovém pásmu.\\
\\
Dalším krokem byl návrh DPS (sekce \ref{s:PRAK}) pro bikvady a PP 2. řádu. Obvod bude prakticky realizován a odzkoušen. Pro lepší návrh by bylo vhodné analyzovat výslednou strukturu popsanou přenosy gyrátorů (sekce \ref{s:MAPLE}) a získat z ní zapojení s~OTA, což by minimalizovalo počet OTA ve~struktuře.
\newpage
\begin{thebibliography}{999}
\bibitem{1}
KAŠPER, Ladislav. \textit{Návrh kmitočtového filtru} [online]. Ostrava, 2012 [cit. 2019-04-28]. Dostupné z: \url{https://dspace.vsb.cz/bitstream/handle/10084/92901/KAS279_FEI_N2647_2601T013_2012.pdf?sequence=1&isAllowed=y}. Diplomová práce. VŠB-TU Ostrava, FEI. Strana 18/69.
\bibitem{2}
\textit{High-pass filtering pre-processing before computing audio features}. Stack Exchange Inc [online]. 2019 [cit. 2019-04-22]. Dostupné z: \url{https://dsp.stackexchange.com/questions/27586/high-pass-filtering-pre-processing-before-computing-audio-features}
\bibitem{3}
MARTINEK, Pravoslav, Petr BOREŠ a Jiří HOSPODKA. \textit{Elektrické filtry}. Praha: Vydavatelství ČVUT, 2003. ISBN 80-01-02765-1.
\bibitem{4}
MICHAL, Vratislav. \textit{Vybrané vlastnosti obvodů pracujících v proudovém módu a napěťovém módu} [online]. Brno, 2017 [cit. 2019-03-30]. Dostupné z: \url{https://docplayer.cz/43256146-Vybrane-vlastnosti-obvodu-pracujicich-v-proudovem-modu-a-napetovem-modu.html}. Článek. Brno University of Technology. Strana 5/6.
\bibitem{5}
HOSPODKA, Jiří. \textit{Úvod do analogových filtrů} [online]. Praha, 2018 [cit. 2019-03-30]. Dostupné z: \url{https://moodle.fel.cvut.cz/course/view.php?id=1434}. Přednáška. ČVUT FEL. Pořadě slide 24/41, 21/41.
\bibitem{6}
SMITH, K.C., SEDRA, A.S. \textit{The current conveyor: a new circuit building block}. IEEE Proc. CAS, 1968,
vol. 56, no. 3, pp. 1368-1369.
\bibitem{7}
SMITH, K.C., SEDRA, A.S. \textit{A second generation current conveyor and its application}. IEEE Trans.,
1970, CT-17, pp. 132-134.
\bibitem{8}
SHAKTOUR, Mahmoud. \textit{Nekonvenční obvodové prvky pro návrh příčkových filtrů} [online]. Brno, 2010 [cit. 2019-10-25]. Dostupné z: \url{https://www.vutbr.cz/www_base/zav_prace_soubor_verejne.php?file_id=35975}. Disertační práce. Vysoké učení technické v Brně. Vedoucí práce Dalibor Biolek. Strana 8/39 - Obrázek 3-1 (a).
\bibitem{9}
\textit{Transconductance Amplifiers} [online]. 2019 [cit. 2019-03-30]. Dostupné z: \url{https://cz.mouser.com/Semiconductors/Integrated-Circuits-ICs/Amplifier-ICs/Transconductance-Amplifiers/_/N-6j73l?P=1y95od0}
\bibitem{10}
LM13700: Dual Operational Transconductance Amplifiers With Linearizing Diodes and Buffers. In: \textit{Texas Instruments} [online]. Dallas, Texas: Texas Instruments Incorporated, 2018 [cit. 2019-03-30]. Dostupné z: \url{www.ti.com/lit/ds/symlink/lm13700.pdf} Strana 1/37. Strana 9/37 - Obrázek 16.
\bibitem{11}
Low-pass filter. In: \textit{Wikipedia: the free encyclopedia} [online]. San Francisco (CA): Wikimedia Foundation, 2001- [cit. 2019-03-30]. Dostupné z: \url{https://en.wikipedia.org/wiki/Low-pass_filter}
\bibitem{12}
SCHAUMANN, Rolf a Mac E. Van VALKENBURG. \textit{Design of Analog Filters}. New York: Oxford University Press, 2001. ISBN 0195118774. Pořadě obrázek 4-13, 4-36 a),b).
\bibitem{13}
RAMSDEN, Ed. \textit{An Introduction to Analog Filters}. Sensors Online [online]. 3 Speen Street, Suite 300, Framingham, MA 01701: Questex, 2019, 1/7/2001 [cit. 2019-05-18]. Dostupné z: \url{https://www.sensorsmag.com/components/introduction-to-analog-filters}
\bibitem{14}
VEDRAL, Josef a Jakub SVATOŠ. \textit{Zpracování a digitalizace analogových signálů v měřící technice}. Praha: Česká technika - nakladatelství ČVUT, 2018. ISBN 978-80-01-06424-5. Strana 136, Obrázek 5.3.9, 5.3.10.
\end{thebibliography}
\end{document}
