%
% Author:  Klára Pacalová
% E-mail:  pacalkla@fel.cvut.cz
% Date:    24.02.2018 11:03
%
\documentclass[twoside]{article}
\usepackage[a4paper]{geometry}
\geometry{verbose,tmargin=2.5cm,bmargin=2cm,lmargin=2cm,rmargin=2cm}
\usepackage{fancyhdr}
\pagestyle{fancy}
\usepackage[utf8]{inputenc}
\usepackage[czech]{babel}
\usepackage{siunitx}
\usepackage{listings}
\usepackage{graphicx} %graphics files inclusion
\usepackage{amsmath} %advanced maths
\usepackage{amssymb} %additional math symbols
% nastavení pisma a češtiny
\usepackage{lmodern}
\usepackage[T1]{fontenc}
\usepackage{float}

% odkazy
\usepackage{url}

% vícesloupcové tabulky
\usepackage{multirow}
\usepackage{listings}

% vnořené popisky obrázků
\usepackage{subcaption}

% automatická konverze EPS 
\usepackage{graphicx} 
\usepackage{epstopdf}

% odkazy a záložky
\usepackage[unicode=true, bookmarks=true,bookmarksnumbered=true,
bookmarksopen=false, breaklinks=false,pdfborder={0 0 0},
pdfpagemode=UseNone,backref=false,colorlinks=true] {hyperref}

% Poznámky při překladu
\usepackage{xkeyval}	% Inline todonotes
\usepackage[textsize = footnotesize]{todonotes}
\presetkeys{todonotes}{inline}{}

% Zacni sekci slovem ukol

% enumerate zacina s pismenem
\renewcommand{\theenumi}{\alph{enumi}}

% smaz aktualni page layout
\fancyhf{}
% zahlavi
\usepackage{titling}
\fancyhf[HC]{\thetitle}
\fancyhf[HLE,HRO]{\theauthor}
\fancyhf[HRE,HLO]{\today}
 %zapati
\fancyhf[FLE,FRO]{\thepage}

% údaje o autorovi
\title{Analogový přeladitelný filtr se zesilovači OTA}
\author{Klára Pacalová}
\date{\today}

\begin{document}
\definecolor{MapleBlue}{rgb}{0,0,1}
\def\MapleOutput#1{{\begin{center}\begin{math}\color{MapleBlue}{#1}\end{math}\end{center}}}

\maketitle

% ---------------------------------
% ---------------------------------
% název sekce je generován automaticky jako: Úkol X
\section{Typy filtrů a jejich aplikace}
Filtry jsou určeny k potlačení nebo zvýraznění určité části kmitočtového spektra signálu. Jsou to obvody s kmitočtově závislou přenosovou funkcí (pro napěťový přenos $H_s(j \omega) = \frac{U_{out}(j \omega)}{U_{in}(j \omega)}$). Základní rozdělení je na dolní propust (\textit{low-pass} - LP), horní propust(\textit{high-pass} - HP), pásmovou propust (\textit{band-pass} - BP) a pásmovou zádrž (\textit{band-stop} - BS). \\
Dolní propust nepropouští na výstup vstupní signál nad frekvencí $f_s$, signál v propustném pásmu zůstává beze změny nebo zesílený. Základní pasivní dvojbranné zapojení je ke vstupu sériově zapojený rezistor a k této větvi paralelně kapacitor. Tento RC člen se zvyšující se frekvencí snižuje svou vstupní impedanci. Přenosová funkce má nulu v nekonečnu a pól v levé polorovině s-roviny. Ideální integrátor má pól v nule. \\
Horní propust nepropouští signály o nízkých frekvencích. Nejjednodušší zapojení je RC člen, kdy kapacitor je zapojen sériově se zdrojem a k této větvi paralelně rezistor. Pro toto zapojení reaktance kapacitoru se zvyšující se frekvencí klesá. Přenosová funkce ideálního derivátoru má pól v nekonečnu a nulu v nule. Horní propust má nulu v nule a pól v levé polorovině s-roviny.\\
Pásmová propust propouští pásmo určené dvěma kmitočty. Pasivní pásmové propusti nedosahují účinnosti větší než 1. Jsou složeny z integračního článku (RC - dolní propust) a derivačního článku (CR - horní propust)\\
Pásmová zádrž nepropouští kmitočty pásma definovaného dvěma kmitočty. Pasivní zapojení je složeno ze dvou rezistorů a kapacitorů. Má vždy ztrátový přenos.
\begin{figure}[H]
\centering
\includegraphics[scale=0.55]{tolerancnischemata.png}
\caption{Toleranční schéma pro a) dolní propust (LP), b) horní propust (HP), c) pásmovou propust (BP) a d) pásmovou zádrž (BS)\cite{1}}
\end{figure}
\noindent Filtry se používají k redukci nežádoucích frekvencí např. pro efektivní reprodukci zvuku reproduktory, k redkci okolního rušení např. vysílače blokují harmonické frekvence, které interferují, jako filtry v obvodech rekonstrukce signálů u D/A převodníků, nebo jako anti-aliasing filtry např předvzorkování u A/D převodníku).\\\\
Obecná přenosová funkce filtru typu dolní propust je
\begin{align}
H(j\omega) = \frac{H_0}{\sum_{i=1}^{n} 1 + a_i s + b_i s^2},
\end{align}
kde $n$ je řád filtru.\\
Obecná přenosová funkce filtru typu horní propust je
\begin{align}
H(j\omega) = \frac{H_{\infty}}{\sum_{i=1}^{n} 1 + \frac{a_i}{s} + \frac{b_i}{s^2}},
\end{align}
kde $n$ je řád filtru.\\
Podle rozložení nul a pólů jmenovatele rozlišujeme různé aproximace. Koeficienty filtru $a_i, b_i$ určují zesílení v propustném pásmu. Činitel jakosti je definován jako $Q = \frac{\sqrt{b_i}}{a_i}$. Čím větší $Q$ je obdrženo, tím spíš bude filtr nestabilní.
\begin{figure}[H]
\centering
\includegraphics[scale=0.3]{LGA98.png}
\caption{Typy aproximací (LP)\cite{2}}
\end{figure}
\subsection{Butterworthova aproximace}
Butterworthova má maximálně plochou amplitudovou charakteristiku v propustném pásmu. Frekvenční charakteristika má sklon daný počtem pólů a pro její posouzení je užíváno skupinové zpoždění (derivace fáze podle frekvence). Pro Butterworthovu aproximaci je skupinové zpoždění nezvlněné v propustném pásmu. Přechodová charakteristika má mírný překmit, zvyšující se s řádem filtru. Zesílení $G(\omega)$ je kmitočtově závislé a odpovídá absolutní hodnotě přenosové funkce $H(j\omega)$.
\begin{align*}
G(\omega) = |H(j\omega)| = \frac{1}{\sqrt{1 + \epsilon ^2 \frac{\omega}{\omega _c}^{2n}}},
\end{align*}
kde $\epsilon$ je poměrné zvlnění kmitočtové charakteristiky v propustném pásmu (\textit{faktor zvlnění}), $n$ je řád filtru a $\omega _c$ mezní frekvence. Mezní frekvence je definována jako frekvence, která nastává při útlumu -3 dB. Pro $\omega _c = 1$ je faktor zvlnění $\epsilon = 1$. 
\subsection{Čebyševova aproximace}
Čebyševova aproximace má strmější pokles, což vede k užití nižšího řádu filtru. Zato má ale zvlněnou frekvenční charakteristiku v propustném pásmu. 
\subsubsection{Typ I}
Vyjádření modulové charakteristiky pro tuto aproximaci je dáno jako
\begin{align}
G(\omega) = |H(j\omega)| = \frac{1}{\sqrt{1 + \epsilon ^2 T_n ^2 \frac{\omega}{\omega _c}^{2n}}},
\end{align}
kde $T_n$ je Čebyševův polynom, $\epsilon$ je poměrné zvlnění, $n$ je řád filtru a $\omega _c$ mezní frekvence. Čebyševův polynom je definován vztahem $2 \omega ^2 - 1$ pro $n = 2$. Obecně jsou to kořeny Chebyshevových diferenciálních rovnic
\begin{align}
(1 - x^2)y" - xy' + n^2y = 0\\
(1 - x^2)y" - 3xy' + n(n+2)y = 0.
\end{align}
\subsubsection{Typ II}
Typ II je nazýván také jako inverzní Čebeševova aproximace. V praxi není příliš používaný, jelikož nemá tak rychlý pokles jako typ I a k jeho realizaci je třeba více prvků. Nemá zvlnění v propustném pásmu, zato v zádržném ano. Zesílení je definováno jako
\begin{align}
G(\omega, \omega _c) = \frac{1}{\sqrt{1 + \frac{1}{\epsilon ^2 T_n ^2 \frac{\omega _c}{\omega}^{2n}}}},
\end{align}
kde $T_n$ je Čebyševův polynom, $\epsilon$ je poměrné zvlnění, $n$ je řád filtru a $\omega _c$ mezní frekvence.
\subsection{Besselova aproximace}
Besselova aproximace se používá v telekomunikační technice v případech, kdy je požadováno zachování tvaru signálu. Amplitudová charakteristika v nepropustném pásmu je velmi plochá. Koeficienty polynomu jsou zvoleny tak, aby fázová charakteristika v pásmu okolo kritické frekvence byla maximálně lineární. Nevýhodou je poměrně malá strmost modulové charakteristiky. Ta je pro Besselovu aproximaci je dána vztahem
\begin{align}
G(\omega) = |H(j\omega)| = \frac{\Theta _n(0)}{\Theta _n(\frac{j\omega}{\omega _c})},
\end{align}
kde $\Phi _n$ je Besselův polynom a $\omega _c$ mezní frekvence. Besselův polynom je definován součtem řady (Grosswald 1978, Berg 2000)
\begin{align}
\Theta _n (x) = x^n y_n (\frac{1}{x}) = \sum_{k=0}^{n}\frac{(n+k)!}{(n-k)!k!}\frac{x^{n-k}}{2^k}.
\end{align}
Pro filtr druhého řádu platí
\begin{align}
G(\omega) = |H(j\omega)| = \frac{3}{\sqrt{\omega ^4 + 3\omega ^2 + 9}}.
\end{align}
\subsection{Cauerova (eliptická) aproximace}
\noindent Cauerova aproximace (eliptická) má nejstrmější pokles, při jejím užití jsou voleny nižší řády filtru. Pokud se zvlnění v zádržném pásmu blíží nule, filtr se stává Čebyševovým (výše zmíněný - typ I). Opačně je tomu v propustném pásmu - přiblížením k nule se filtr stává inverzním Čebyševovým (typ II).  Pokud se obě hodnoty zvlnění blíží k nule, filtr se stává Butterworthovým. Kmitočtová charakteristika je dána vztahem
\begin{align}
G(\omega) = |H(j\omega)| = \frac{1}{\sqrt{1 + \epsilon ^2 R_n ^2(\zeta, \frac{\omega}{\omega _c})}},
\end{align}
kde $\epsilon$ je faktor zvlnění, $R_n$ eliptická racionální funkce n-tého řádu, $\zeta$ selektivní faktor a $\omega _c$ mezní frekvence. Pokud pro selektivní faktor platí $\zeta \rightarrow \infty$, filtr se stává Čebyševovým (typ I). 
\section{Transkonduktanční zesilovače (OTA)}
V telekomunikacích se používají filtry v rozsahu kmitočtů desítek až stovek megahertz, v bezdrátové komunikaci až v řádu gigahertz. Běžné RC filtry by neměly být užívány ve frekvenčním rozsahu nad 5-10$\%$ $\omega _c$ - tedy v tomto rozsahu používaném v telekomunikačních technologiích nemají předvídatelné průběhy. Krom toho ve spínačích CMOS, kde rezistory běžně nejsou dostupné, jsou potřeba zesilovače s velkou šířkou pásma a zároveň vysokým zesílením. Dodržení těchto požadavků je náročné a drahé. Dalším extrémem pro analogové integrované filtry jsou telefonní linky, kde jsou kmitočtové rozsahy sice nízké, ale je požadována nízká cena a vysoká přesnost.\\
Pro nízké frekvence se ke splnění těchto požadavků používají obvody se spínanými kapacitory (SC). Přepínaný kapacitor se chová jako rezistor, tudíž časová konstanta RC je definována poměrem kapacitorů a hodinovou (CLK) frekvencí, se kterou jsou přepínány. Pro vysokofrekvenční aplikace (až v řádu gigahertz) se používají MOSFET-C filtry.\\
Další z možných prvků, které jsou dostupné jak pro nízkofrekvenční aplikace, tak pro kmitočtový rozsah stovek megahertz, jsou transkonduktanční zesilovače.\\\\
Transkonduktanční zesilovače (označují se též jako OTA (\textit{Operational Transconductance Amplifiers}) jsou napětím řízené zesilovače s proudovým výstupem - zdroje proudu
\begin{align}
i_{out} = g_m(u_+ - u_-),
\end{align}
kde $u_+$ a $u_-$ jsou napětí invertujícího a neinvertujícího vstupu.  Transkonduktance je řízena externím proudem $I_{ABC}$ (\textit{Bias Current}). Ideální OTA má kmitočtově nezávislou transkonduktanci $g_m$ (na rozdíl od reálného, který je kmitočtově závislý).
\begin{figure}[H]
\centering
\includegraphics[scale=0.7]{image7.png}
\caption{OTA - schematické značky \cite{3}}
\end{figure}
\begin{figure}[H]
\centering
\includegraphics[scale=0.6]{gmrc.png}
\caption{Linearizovaný model reálného OTA \cite{4}}
\end{figure}
\noindent Připojením zátěže $R_z$ na výstup bylo získáno napětí naprázdno
\begin{align}
u_{out} = R_zg_m(u_+ - u_-) = G_0(u_+ - u_-),
\end{align}
kde $G_0$ je zesílení. Ze vztahu (2) plyne, že zesílení je konečné a mezi vstupy je nenulové napětí. Připojením kondenzátoru jako zátěže byl získán bezeztrátový integrátor s přenosem
\begin{align}
H(s) &= \frac{v_2}{v_1} = \frac{g_m}{sC} \\
v_0(t) &= \frac{1}{C}\int i(t)dt = \frac{1}{C}\int g_mv_1(t)dt.
\end{align}
\begin{figure}[H]
\centering
\includegraphics[scale=0.5]{otaintegrator.png}
\caption{OTA-C \cite{4}}
\end{figure}
\noindent Toto zapojení integrátoru s uzemněným kondenzátorem se označuje jako OTA-C.\\
\\
Ztrátový integrátor lze utvořit sériovým zapojením dalšího OTA jako odporu se zápornou zpětnou vazbou. Rozdíl mezi ideálním a ztrátovým integrátorem lze pozorovat i v modulové charakteristice - pro ztrátový je konstantní a pak teprve lineárně klesá se sklonem -20 dB/dek.
\begin{align}
v_0(t) = \frac{g_{m1}}{sC + g_{m2}}(v_1^+ - v_{1}^-)
\end{align}
\begin{figure}[H]
\centering
\includegraphics[scale=0.5]{damp.png}
\caption{Ztrátový OTA-C \cite{4}}
\end{figure}
\section{Integrované obvody s OTA zesilovači}
Integrované obvody se vyrábí buď s jedním nebo dvěma zesilovači v pouzdře. Varianty s jedním operačním zesilovačem jsou např. OPA615, OPA860 a novější OPA861. Všechny součástky s jedním OZ mají velkou šířku pásma (v řádech stovek MHz), cenově vychází na 75-280 Kč. Integrované obvody s dvěma OZ v pouzdře mají užší šířku pásma (2 MHz), menší rychlost přeběhu (50 V/$\mu$s), mnohem menší výstupní proud (650 $\mu$A) i offset vstupního napětí a operují při cca 4x nižších proudech. Cenové rozpětí je 25-65 Kč.
\renewcommand{\arraystretch}{1.5}
\begin{table}[H]
\scalebox{0.9}{%
  \begin{tabular}{ | c | >{\centering\arraybackslash}p{2cm}| >{\centering\arraybackslash}p{1.5cm} | >{\centering\arraybackslash}p{1.5cm} | >{\centering\arraybackslash}p{1.25cm} | >{\centering\arraybackslash}p{1.5cm} | >{\centering\arraybackslash}p{1.75cm} | >{\centering\arraybackslash}p{2cm} | >{\centering\arraybackslash}p{1.75cm} |}
    \hline
      & GBP - Gain Bandwidth Product & SR - Slew Rate & Output Current per Channel & $I_b$ - Input Bias Current & $V_{os}$ - Input Offset Voltage & Operating Supply Current & Forward Transconductance Min & Supply Voltage\\ \hline
    OPA615 & 710 MHz & 2.5 kV/$\mu$s & 5 mA & 3 $\mu$A & 40 mV & 13 mA & 65 mA/V & 8-12.4 V \\ \hline
    OPA860 & 470 MHz & 3.5 kV/$\mu$s & 15 mA & 5 $\mu$A & 12 mV & 11.2 mA & 80 mA/V & 5-13 V \\ \hline
    OPA861 & 400 MHz & 900 V/$\mu$s & 15 mA & 1 $\mu$A & 12 mV & 5.4 mA & 65 mA/V & 4-12.6 V  \\
    \hline
  \end{tabular}}
  \caption{\label{tab:Porovnání integrovaných obvodů s jedním OTA}orovnání integrovaných obvodů s jedním OTA \cite{5}}
  \end{table}
\begin{center}
\begin{table}[H]
\scalebox{0.9}{%
  \begin{tabular}{ | c | >{\centering\arraybackslash}p{2cm}| >{\centering\arraybackslash}p{1.5cm} | >{\centering\arraybackslash}p{1.5cm} | >{\centering\arraybackslash}p{1.25cm} | >{\centering\arraybackslash}p{1.5cm} | >{\centering\arraybackslash}p{1.75cm} | >{\centering\arraybackslash}p{2cm} | >{\centering\arraybackslash}p{1.75cm} |}
    \hline
      & GBP - Gain Bandwidth Product & SR  - Slew Rate & Output Current per Channel & $I_b$ - Input Bias Current & $V_{os}$ - Input Offset Voltage & Operating Supply Current & Forward Transconductance - Min & Supply Voltage\\ \hline
    LM13700 & 2 MHz & 50 V/$\mu$s & 650 $\mu$A & 5 $\mu$A & 4 mV & 1.3 mA & 6700 $\mu$S & 10-36 V \\ \hline
    NE5517 & 2 MHz & 50 V/$\mu$s & 650 $\mu$A & 5 $\mu$A & 5 mV & 2.6 mA & 5400 $\mu$S & 4-44 V \\ \hline
    AU5517 & 2 MHz & 50 V/$\mu$s & 650 $\mu$A & 5 $\mu$A & 5 mV & 2.6 mA & 5400 $\mu$S & 4-44 V  \\ \hline
    NJM13600 & 2 MHz & 50 V/$\mu$s & 650 $\mu$A & 5 $\mu$A & 5 mV & 2.6 mA & 6700 $\mu$S & 36 V  \\ \hline
    NJM13700 & 2 MHz & 50 V/$\mu$s & 650 $\mu$A & 5 $\mu$A & 4 mV & 2.6 mA & 6700 $\mu$S & 36 V  \\ \hline
  \end{tabular}}
  \caption{\label{tab:Porovnání integrovaných obvodů se dvěma OTA}Porovnání integrovaných obvodů se dvěma OTA \cite{5}}
  \end{table}
\end{center}
\noindent Pro realizaci přeladitelného filtru byl zvolen LM13700 s dvěma OZ.
\begin{figure}[H]
\centering
\includegraphics[scale=0.55]{image6.png}
\caption{Konfigurace pinů na LM13700M \cite{6}}
\end{figure}
\noindent Vnitřní zapojení LM13700 na obrázku 8 obsahuje symetrický rozdílový stupeň (tranzistory Q4, Q5), který je napájen řízeným zdrojem proudu s tranzistorem Q2. Dvojice diod a tranzistorů tvoří proudová zrcadla (\textit{Current Mirror}) - referenční proud tekoucí v jedné větvi obvodu se "zrcadlí" v jeho druhé větvi. Principiálně jsou to zdroje proudu řízené proudem. 
\begin{figure}[H]
\centering
\includegraphics[scale=0.75]{image5.png}
\caption{Vnitřní chéma OTA \cite{6}}
\end{figure}
\section{Odvození}
Náhradní obvod, ze kterého bude spočítána přenosová funkce pro přenos filtru druhého řádu, popisuje obrázek 9.
\begin{figure}[H]
\centering
\includegraphics[scale=0.15]{RLC_low-pass.png}
\caption{Dolní propust 2. řádu (RLC obvod) \cite{7}}
\end{figure}
\noindent Přenos obvodu byl vyjádřen jako
\begin{align}
H(s) = \frac{U_{out}}{U_{in}} = \frac{Z_2}{Z_1},
\end{align}
kde $Z_1 = sL$ a $Z_2 = \frac{\frac{R}{sC}}{R + \frac{1}{sC}}$. Tedy
\begin{align}
H(s) = \frac{\frac{\frac{R}{sC}}{R + \frac{1}{sC}}}{sL + \frac{\frac{R}{sC}}{R + \frac{1}{sC}}}.
\end{align}
Elementárními algebraickými úpravami a následným vynásobením členem $\frac{1}{LRC}$ byl získán výsledný přenos.
\begin{align}
H(s) = \frac{R}{s^2LRC + sL + R} = \frac{\frac{1}{LC}}{s^2 + \frac{s}{RC} + \frac{1}{LC}}.
\end{align}
\noindent Pro ideální OTA zesilovač (vstupní i výstupní impedance nulové) je možno odpor nahradit obvodem s uzemněným neinvertujícím vstupem a zpětnou vazbou z invertujícího vstupu na výstup a to hodnotou
\begin{align}
R_{in} = \frac{1}{g_{m1}},
\end{align}
kde $g_{m1}$ označuje transkonduktanci zesilovače. Prohození invertujícího a neinvertujícího vstupu vede na opačnou polaritu.
\begin{figure}[H]
\centering
\includegraphics[scale=0.7]{image10.png}
\caption{Obvod pro simulaci uzemněného rezistoru \cite{8}}
\end{figure}
\noindent Pro nahrazení indukčnosti o impedanci $Z_L = \frac{1}{sC}$ lze použít obvod s třemi OTA. Uzemněny jsou invertující vstup prvního OTA a neinvertující druhého. Použita je zpětná vazba z výstupu na neinvertující vstup prvního OTA. Propojení výstupu prvního OTA na invertující vstup druhého OTA je realizován přes uzemněný kapacitor. \\
Vyjádřením napětí a proudů v obvodu bylo získáno napětí na kapacitoru a vstupní proud
\begin{align}
V_C &= \frac{g_{m1}}{sC}V_1 \\
I_1 &= g_{m2}V_C = \frac{g_{m1}g_{m2}}{sC}V_1.
\end{align}
Výsledná indukčnost - impedance vstupu byla vyjádřena vztahem (11).
\begin{align}
Z_{in}(s) = \frac{V_1}{I_1} = s\frac{C}{g_{m1}g_{m2}}
\end{align}
\noindent Byl obdržen induktor o hodnotě
\begin{align}
L = \frac{C}{g_{m1}g_{m2}}.
\end{align}
\begin{figure}[H]
\centering
\includegraphics[scale=0.55]{image13.png}
\caption{Obvod pro simulaci indukčnosti \cite{8}}
\end{figure}
\noindent Pro uzemněnou indukčnosti o impedanci $Z_L = \frac{1}{sC}$ byl použit obvod na obrázku 12. Vyjádřením napětí a proudů v obvodu bylo získáno napětí na kapacitoru a vstupní proud
\begin{align}
V_C &= \frac{g_{m1}}{sC}V_1 \\
I_1 &= g_{m2}V_C = \frac{g_{m1}g_{m2}}{sC}V_1.
\end{align}
Výsledná indukčnost - impedance vstupu byla vyjádřena vztahem (16).
\begin{align}
Z_{in}(s) = \frac{V_1}{I_1} = s\frac{C}{g_{m1}g_{m2}}
\end{align}
\begin{figure}[H]
\centering
\includegraphics[scale=1]{image12.png}
\caption{Obvod pro simulaci uzemněné indukčnosti pro $g_{m1} = g_{m2}$\cite{8}}
\end{figure}
\noindent Nyní je možno za odpor a indukčnost dosadit do vztahu (8). Byly uvažovány kapacitory o stejné hodnotě C.
\begin{align}
H(s) = \frac{\frac{1}{\frac{C^2}{g_{m1}g_{m2}}}}{s^2 + \frac{s}{\frac{C}{g_{m2}}} + \frac{1}{\frac{C^2}{g_{m1}g_{m2}}}} = \frac{\frac{g_{m1}g_{m2}}{C^2}}{s^2 + \frac{sg_{m2}}{C} + \frac{g_{m1}g_{m2}}{C^2}} = \frac{g_{m1}g_{m2}}{s^2C^2 + sg_{m2}C + g_{m1}g_{m2}}.
\end{align}
Porovnáním jmenovatele se jmenovatelem přenosu filtru 2. řádu byl obdržen vztah
\begin{align}
s^2 + s\frac{\omega _c}{Q} + \omega _c^2 &= s^2C^2 + sg_{m2}C + g_{m1}g_{m2}\\
s^2 + s\frac{\omega _c}{Q} + \omega _c^2 &= s^2 + \frac{sg_{m2}}{C} + \frac{g_{m1}g_{m2}}{C^2}.
\end{align}
Z tohoto vztahu byl vyjádřen mezní kmitočet jako 
\begin{align}
\omega _c^2 &= \frac{g_{m1}g_{m2}}{C^2} \\
\omega _c &= \sqrt{\frac{g_{m1}g_{m2}}{C^2}}
\end{align}
a činitel jakosti dosazením za $\omega _c$
\begin{align}
Q = \frac{\omega _c}{\frac{g_{m2}}{C}} = \sqrt{\frac{g_{m1}}{g_{m2}}}.
\end{align}
Pokud navíc byly uvažovány stejné transkonduktance $g_{m1},g_{m2} = g_m$, byl obdržen výsledek
\begin{align}
\omega _c &= \sqrt{\frac{g_m^2}{C^2}},\\
Q &= \sqrt{1} = 1.
\end{align}
\section{Dolní propust 2. řádu}
Dolní propust druhého řádu má přenos v nekonečnu nulový $H_{\infty} = 0$. Přenosová funkce je
\begin{align}
H(j\omega) = \frac{H_0 \omega_c ^2}{(j\omega)^2 + \frac{\omega _c}{Q}(j\omega) + \omega _c ^2}.
\end{align}
Obvodová simulace byla realizována v programu Multisim. Zapojení dvou OTA-C v sérii vede na dolní propust druhého řádu. Bylo zvoleno symetrické napájení OZ $V_{DD},V_{SS} = \pm 15$ V. Regulací vstupního proudu je ovlivňován pracovní bod obvodu (mezní kmitočet). Vstupní externí proud $I_{ABC} = 0.5$ $\mu$A byl zvolen tak, aby byl obdržen mezní kmitočet cca 100 kHz. Externím proudem $I_{ABC} \in$ $<5$ $\mu$A ; 500 $\mu$A> je výrobcem garantováno minimální výstupní napětí $U_{OUT} = \pm 12$ V, standardně $V_{peak 1} = 14.2$ V a $V_{peak 2} = -14.4$ V. Při výstupním napětí v tomto intervalu je šum vzhledem k signálu zanedbatelný a nezkreslí výsledky simulace.\\
\begin{figure}[H]
\centering
\includegraphics[scale=0.75]{1503.png}
\caption{Schéma zapojení dolní propusti 2. řádu}
\end{figure}
\begin{figure}[H]
\centering
\includegraphics[scale=0.75]{15032.png}
\caption{Amplitudová a fázová charakteristika dolní propusti 2. řádu}
\end{figure}
\noindent Obvod lze realizovat i zapojením indukčnosti(náhradní schéma pro OTA - Obrázek 11), odporu (náhradní schéma - Obrázek 10) a uzemněného kapacitoru. 
\begin{figure}[H]
\centering
\includegraphics[scale=0.8]{171.png}
\caption{RLC obvod \cite{9}}
\end{figure}
\noindent Mezní frekvence a činitel jakosti tohoto obvodu byly spočítány jako 
\begin{align}
\omega _c &= \sqrt{\frac{1}{LC}}\\
Q &= \frac{L\sqrt{\frac{C}{L}}}{RC}.
\end{align}
Zapojení obvodu v Multisimu ilustruje Obrázek 15.
\begin{figure}[H]
\centering
\includegraphics[scale=0.7]{1707.png}
\caption{Schéma zapojení dolní propusti 2. řádu}
\end{figure}
\begin{figure}[H]
\centering
\includegraphics[scale=0.75]{17072.png}
\caption{Amplitudová a fázová charakteristika dolní propusti 2. řádu}
\end{figure}
\noindent Lze použít i zapojení z kapitoly 4 s uzemněným kapacitorem a odporem, avšak při této realizaci dochází v amplitudové charakteristice k překmitu. Proto bylo zvoleno řešení zmíněné výše.
\section{Dolní propust čtvrtého řádu - kaskádně}
Kaskádní zapojení je realizováno násobením sériově zapojených bloků.
\begin{figure}[H]
\centering
\includegraphics[scale=0.4]{schemata.png}
\caption{Kaskádní zapojení \cite{4}}
\end{figure}
Přenosové funkce jednotlivých bloků se násobí
\begin{align}
H_k(j\omega) = \frac{U_k (j\omega)}{U_{k-1}(j\omega)}.
\end{align}
Přenos posledního bloku je dán vztahem
\begin{align}
H_{1 \rightarrow k}(j\omega) = \frac{U_k (j\omega)}{U_{in}(j\omega)} = \sum_{n=1}^{k} H_n(j\omega).
\end{align}
Kaskádním zapojením dvou dolních propusti ze sekce 5 byl obdržen filtr 4. řádu s poklesem -80 dB/dek. 
\begin{figure}[H]
\centering
\includegraphics[scale=0.75]{1506.png}
\caption{Schéma kaskádního zapojení dolní propusti 4. řádu}
\end{figure}
\begin{figure}[H]
\centering
\includegraphics[scale=0.75]{15062.png}
\caption{Amplitudová a fázová charakteristika káskádního zapojení dolní propusti 4. řádu}
\end{figure}
\section{Pásmová propust}
Horní propust druhého řádu má přenos v nule nulový $H_{0} = 0$. Přenosová funkce je
\begin{align}
H(j\omega) = \frac{H_{\infty} (j\omega) ^2}{(j\omega)^2 + \frac{\omega _c}{Q}(j\omega) + \omega _c ^2}.
\end{align}
\noindent Nejprve byla získána horní propust kaskádním zapojením dvou RC článků.
\begin{figure}[H]
\centering
\includegraphics[scale=0.75]{1606.png}
\caption{Schéma zapojení horní propusti 2. řádu}
\end{figure}
\begin{figure}[H]
\centering
\includegraphics[scale=0.75]{16063.png}
\caption{Amplitudová a fázová charakteristika horní propusti 2. řádu}
\end{figure}
Pásmová propust má přenos v nule i nekonečnu nulový $H_{0} = H_{\infty} = 0$. Přenosová funkce je
\begin{align}
H(j\omega) = \frac{H_{B} \frac{\omega _c}{Q} (j\omega) }{(j\omega)^2 + \frac{\omega _c}{Q}(j\omega) + \omega _c ^2}.
\end{align}
\noindent Následně byla sériovým zapojením dolní a horní propusti 2. řádu obdržena pásmová propust 2. řádu.
\begin{figure}[H]
\centering
\includegraphics[scale=0.75]{16064.png}
\caption{Schéma zapojení pásmové propusti 2. řádu}
\end{figure}
\begin{figure}[H]
\centering
\includegraphics[scale=0.75]{16065.png}
\caption{Amplitudová a fázová charakteristika pásmové propusti 2. řádu}
\end{figure}
\section{Pásmová propust čtvrtého řádu}
Kaskádním zapojením dvou pásmových propustí 2. řádu byl obdržen filtr 4. řádu s poklesem -80 dB/dek. 
\begin{figure}[H]
\centering
\includegraphics[scale=0.6]{kaskadnebandpass1.png}
\caption{Schéma kaskádního zapojení pásmové propusti 4. řádu}
\end{figure}
\begin{figure}[H]
\centering
\includegraphics[scale=0.75]{kaskadnebandpass.png}
\caption{Amplitudová a fázová charakteristika káskádního zapojení pásmové propusti 4. řádu}
\end{figure}
\section{LC filtry}
Pasivní dolní propust je realizována zapojením induktoru ke vstupnímu napětí a k této větvi je následně zapojen paralelně rezistor. Pasivní horní propust má ke vstupu připojený sériově rezistor a poté k této větvi paralelně induktor. \\
K realizaci filtrů vyšších řádů se užívají $\pi$ nebo T články s LC prvky. Při návrhu filtru musí být zohledněn vnitřní odpor zdroje $R_s$ a zatěžovací odpor $R_L$. LC filtry jsou tedy dvojitě zakončeny. Indukčnosti a kapacity prvků se určí z rovnic pro normované kapacity a indukčnosti. Normované hodnoty budou vypočteny pro mezní kmitočet $\omega _c = \frac{1}{\sqrt{LC}}$ a pro zatěžovací odpor $R_L$. Hodnoty prvků lze pro požadovanou aproximaci odečíst z tabulek. \\
\begin{figure}[H]
\centering
\includegraphics[scale=0.1]{piclanky.png}
\caption{Pasivní dolní propust n-tého řády s $\pi$ články \cite{10}}
\end{figure}
\begin{figure}[H]
\centering
\includegraphics[scale=0.08]{tclanky.png}
\caption{Pasivní dolní propust n-tého řády s T články \cite{10}}
\end{figure}
\section{Návrh v Maple}
Byly zvoleny parametry tolerančního schématu 
\MapleOutput{fm = 80000 Hz}
\MapleOutput{delta\_{fp} = 130000 Hz}
\MapleOutput{delta\_{fs} = 300000 Hz}
\MapleOutput{ap = 20 dB}
\MapleOutput{as = 80  dB,}
\noindent kde $fm$ značí geometrický střed propustného pásma [Hz], $delta\_{fp}$ šířku propustného pásma [Hz],
$delta\_{fs} $ šířku nepropustného pásma [Hz], $ap$ maximální útlum v propustném pásmu [dB], $as$ minimální útlum v nepropustném pásmu [dB].
Funkcí $BP22NLP$ byly spočteny spodní a horní hranice nepropustného pásma $f\_s,fs$ a spodní a horní hranice propustného pásma $f\_p,fp$.
\begin{align}
f\_s &= \frac{\sqrt{delta\_{fs}^2+4f\_m ^2}-delta\_{fs}}{2}\\
f\_p &= \frac{\sqrt{delta\_{fp}^2+4f\_m ^2}-delta\_{fp}}{2}\\
fp &= \frac{\sqrt{delta\_{fp}^2+4f\_m ^2}+delta\_{fp}}{2}\\
fs &= \frac{\sqrt{delta\_{fs}^2+4f\_m ^2}+delta\_{fs}}{2}
\end{align}
\MapleOutput{f\_s = 20000 Hz}
\MapleOutput{f\_p = 38077 Hz}
\MapleOutput{fp = 168077 Hz}
\MapleOutput{fs = 320000 Hz}
\noindent Byl obdržen kmitočet hranice nepropustného pásma normované dolní propusti (NDP) $Os$ [1/s].
\MapleOutput{Os = 2.307692 1/s}.
\begin{figure}[H]
\centering
\includegraphics[scale=0.7]{tolsch.png}
\caption{Toleranční schéma navrhované pásmové propusti}
\end{figure}
\begin{thebibliography}{999}
\bibitem{1}
KAŠPER, Ladislav. \textit{Návrh kmitočtového filtru} [online]. Ostrava, 2012 [cit. 2019-04-28]. Dostupné z: \url{https://dspace.vsb.cz/bitstream/handle/10084/92901/KAS279_FEI_N2647_2601T013_2012.pdf?sequence=1&isAllowed=y}. Diplomová práce. VŠB-TU Ostrava, FEI. Strana 18/69.
\bibitem{2}
\textit{High-pass filtering pre-processing before computing audio features}. Stack Exchange Inc [online]. 2019 [cit. 2019-04-22]. Dostupné z: \url{https://dsp.stackexchange.com/questions/27586/high-pass-filtering-pre-processing-before-computing-audio-features}
\bibitem{3}
MICHAL, Vratislav. \textit{Vybrané vlastnosti obvodů pracujících v proudovém módu a napěťovém módu} [online]. Brno, 2017 [cit. 2019-03-30]. Dostupné z: \url{https://docplayer.cz/43256146-Vybrane-vlastnosti-obvodu-pracujicich-v-proudovem-modu-a-napetovem-modu.html}. Článek. Brno University of Technology. Strana 5/6.
\bibitem{4}
HOSPODKA, Jiří. \textit{Úvod do analogových filtrů} [online]. Praha, 2018 [cit. 2019-03-30]. Dostupné z: \url{https://moodle.fel.cvut.cz/course/view.php?id=1434}. Přednáška. ČVUT FEL. Pořadě slide 24/41, 21/41.
\bibitem{5}
\textit{Transconductance Amplifiers} [online]. 2019 [cit. 2019-03-30]. Dostupné z: \url{https://cz.mouser.com/Semiconductors/Integrated-Circuits-ICs/Amplifier-ICs/Transconductance-Amplifiers/_/N-6j73l?P=1y95od0}
\bibitem{6}
LM13700: Dual Operational Transconductance Amplifiers With Linearizing Diodes and Buffers. In: \textit{Texas Instruments} [online]. Dallas, Texas: Texas Instruments Incorporated, 2018 [cit. 2019-03-30]. Dostupné z: \url{www.ti.com/lit/ds/symlink/lm13700.pdf} Strana 1/37. Strana 9/37 - Obrázek 16.
\bibitem{7}
Low-pass filter. In: \textit{Wikipedia: the free encyclopedia} [online]. San Francisco (CA): Wikimedia Foundation, 2001- [cit. 2019-03-30]. Dostupné z: \url{https://en.wikipedia.org/wiki/Low-pass_filter}
\bibitem{8}
SCHAUMANN, Rolf a Mac E. Van VALKENBURG. \textit{Design of Analog Filters}. New York: Oxford University Press, 2001. ISBN 0195118774. Pořadě obrázek 4-13, 4-36 a),b).
\bibitem{9}
WADE, Augustus. Presentation on theme: \textit{Circuits for sensors Ideal OP Amps Basic OP Amp Circuit Blocks} [online]. In: . 2015 [cit. 2019-04-26]. Dostupné z: \url{https://slideplayer.com/slide/4458062} Prezentace. Slide 20/48.
\bibitem{10}
VEDRAL, Josef a Jakub SVATOŠ. \textit{Zpracování a digitalizace analogových signálů v měřící technice}. Praha: Česká technika - nakladatelství ČVUT, 2018. ISBN 978-80-01-06424-5. Strana 136, Obrázek 5.3.9, 5.3.10.
\end{thebibliography}
\end{document}