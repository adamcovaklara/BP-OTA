\noindent Cílem práce je navrhout zapojení analogového přeladitelného filtru se zesilovači OTA. Filtr je zvolen typu pásmová propust 4. řádu s Cauerovou aproximací. K realizaci byl použit LM13700 kvůli dostačující šířce pásma (2 MHz) a cenové dostupnosti. Mezní kmitočet byl zvolen v řádu stovek kHz, což umožňuje využití např. pro přenos rozhlasového vysílání v atmosféře. Tyto dlouhé vlny (30 -- 300 kHz, čemuž odpovídá délka vlny 1 -- 10 km) obtékají nerovnosti a jdou za obzor bez nutnosti odrazu. Dnes na dlouhých vlnách vysílá jen několik národních rozhlasových vysílačů velkých států a pásmo se hlavně využívá pro takové účely, kde je na prvním místě spolehlivost a výhody pozemní vlny. To jsou například frekvenční a časové standardy (DCF77), radiomajáky, případně i komunikace s ponorkami. Střední vlny (525 -- 1705 khz, což odpovídá vlnovým délkám 186 -- 577 m) mají menší dosah a často u nich dochází k jednomu odrazu od atmosféry. Lépe se ohýbají za přírodními překážkami a jsou vhodné pro vysílání v okruhu stovek kilometrů. Simulace pásmové propusti s OTA byla realizována v MultiSimu, výhodou zde byla možnost využití bloku LM13700 bez nutnosti modelovat obvod vstupním diferenčním stupněm a proudovými zrcadly. Syntéza filtru z matematického hlediska byla provedena v Maplu s knihovnami Syntfil a PraCAn vyvinutými katedrou teorie obvodů. K praktické realizaci byl využit KiCad z důvodu multiplatformní podpory (Linux, OS-X, Win). \\
\\
\noindent \textbf{Klíčová slova:} \textit{transkonduktance, OTA, OTA-C, analogový filtr, pásmová propust, dolní propust}\\

\noindent The purpose of this thesis is to design a schematics of an analog filter with a variable cut-off frequency using OTA. Filter to design is specified to be a band-pass of fourth order with Cauer approximation. For the realization was used LM13700 due to its comfortable bandwidth (2 MHz) and price affordability. Cut-off frequency was chosen in the range of hundreds kHz, which can be used i.e. for transmission of radio broadcasting in the atmosphere. These long waves (30 - 300 kHz, which corresponds to a wave length of 1 - 10 km) bypass unevenness and go beyond the horizon without the need for reflection. Today, only a few national radio transmitters of large states are broadcasting on long waves, and the band is mainly used for purposes where the reliability and benefits of terrestrial waves are paramount. These are, for example, frequency and time standards (DCF77), radio beacons, or communication with submarines. Medium waves (525 -- 1705 kHz, which corresponds to wavelengths of 186 -- 577 m) have a smaller radius and often have a single reflection from the atmosphere. They better bend behind natural obstacles and are suitable for broadcasting within hundreds of kilometers. Simulation of bandpass filter with OTA was realized in MultiSim, the advantage was the possibility to use LM13700 block without the need to model the circuit with differential input stage and current mirrors. Mathematically, filter synthesis was performed in Maple with the Syntfil and PraCAn libraries developed by the Department of Circuit Theory. For practical implementation was used KiCad because of its cross-platform support (Linux, OS-X, Win). \\

\noindent \textbf{Klíčová slova:} \textit{transconductance, OTA, OTA-C, analog filter, band-pass, low-pass} \\