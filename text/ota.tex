V telekomunikacích se používají filtry v rozsahu kmitočtů desítek až stovek megahertz, v bezdrátové komunikaci až v řádu gigahertz. Běžné RC filtry by neměly být užívány ve frekvenčním rozsahu nad 5--10 $\%$ $\omega _c$ - tedy v tomto rozsahu používaném v telekomunikačních technologiích nemají předvídatelné průběhy. Krom toho ve spínačích CMOS, kde rezistory běžně nejsou dostupné, jsou potřeba zesilovače s velkou šířkou pásma a zároveň vysokým zesílením. Dodržení těchto požadavků je náročné a drahé. Dalším extrémem pro analogové integrované filtry jsou telefonní linky, kde jsou kmitočtové rozsahy sice nízké, ale je požadována nízká cena a vysoká přesnost.\\
\\
Pro nízké frekvence se ke splnění těchto požadavků používají obvody se spínanými kapacitory (SC). Přepínaný kapacitor se chová jako rezistor, tudíž časová konstanta RC je definována poměrem kapacitorů a hodinovou (CLK) frekvencí, se kterou jsou přepínány. Pro vysokofrekvenční aplikace (až v řádu gigahertz) se používají MOSFET-C filtry.\\
\\
Další z možných prvků, které jsou dostupné jak pro nízkofrekvenční aplikace, tak pro kmitočtový rozsah stovek megahertz, jsou transkonduktanční zesilovače.\\
\\
Transkonduktanční zesilovače (označují se též jako OTA (\textit{Operational Transconductance Amplifiers}) jsou napětím řízené zesilovače s proudovým výstupem - zdroje proudu
\begin{equation}
i_{out} = g_m(u_+ - u_-),
\end{equation}
kde $u_+$ a $u_-$ jsou napětí invertujícího a neinvertujícího vstupu.  Transkonduktance je řízena externím proudem $I_{ABC}$ (\textit{Bias Current}). Ideální OTA má kmitočtově nezávislou transkonduktanci $g_m$ (na rozdíl od reálného, který je kmitočtově závislý).
\begin{figure}[h]
\centering
\includegraphics[scale=0.7]{image7.png}
\caption[OTA - schematické značky]{OTA - schematické značky \cite{4}}
\end{figure}
\begin{figure}[h]
\centering
\includegraphics[scale=0.6]{gmrc.png}
\caption[Linearizovaný model reálného OTA]{Linearizovaný model reálného OTA \cite{5}}
\end{figure}
\noindent Připojením zátěže $R_z$ na výstup bylo získáno napětí naprázdno
\begin{equation}\label{s:vzt}
u_{out} = R_zg_m(u_+ - u_-) = G_0(u_+ - u_-),
\end{equation}
kde $G_0$ je zesílení. Ze vztahu \ref{s:vzt} plyne, že zesílení je konečné a mezi vstupy je nenulové napětí. Připojením kondenzátoru jako zátěže byl získán bezeztrátový integrátor s přenosem
\begin{equation}
H(s) = \frac{v_2}{v_1} = \frac{g_m}{sC}
\end{equation}
\noindent a napětím na výstupu
\begin{equation}
v_0(t) = \frac{1}{C}\int i(t)dt = \frac{1}{C}\int g_mv_1(t)dt.
\end{equation}
\begin{figure}[h]
\centering
\includegraphics[scale=0.5]{otaintegrator.png}
\caption[OTA-C]{OTA-C \cite{5}}
\end{figure}
\noindent Toto zapojení integrátoru s uzemněným kondenzátorem se označuje jako OTA-C.\\
\\
Ztrátový integrátor lze utvořit sériovým zapojením dalšího OTA jako odporu se zápornou zpětnou vazbou. Rozdíl mezi ideálním a ztrátovým integrátorem lze pozorovat i v modulové charakteristice - pro ztrátový je konstantní a pak teprve lineárně klesá se sklonem -20 dB/dek.
\begin{equation}
v_0(t) = \frac{g_{m1}}{sC + g_{m2}}(v_1^+ - v_{1}^-)
\end{equation}
\begin{figure}[h]
\centering
\includegraphics[scale=0.5]{damp.png}
\caption[Ztrátový OTA-C]{Ztrátový OTA-C \cite{5}}
\end{figure}
\subsection{Current Conveyor of Second Generation (CCII) s OTA}
Jeden z nejzákladnějších bloků v oblasti analogov obvodů v proudovém módu je \textit{current conveyor (CC)}. Princip CC první generace byl popsán v roce 1968 (K. C. Smith, A. S. Sedra \cite{6}). \textit{CCI} byl následně nahrazen univerzálnější druhou generací v roce 1970 (\textit{CCII})\cite{7}. Obvody s CC se používaly především v zapojeních s bipolárními tranzistory kvůli jejich vysoké transkonduktanci (v porovnání s CMOS). Jsou to operační zesilovače s proudovou zpětnou vazbou (např. MAX477, MAX4112). \textit{Current conveyors} jsou používány ve vysokofrekvenčních obvodech, kde je problematické použití běžných operačních zesilovačů, protože jsou limitovány násobkem šířky pásma a zesílení (\textit{gain-bandwidth product}). Je to struktura s třemi vstupy.
\begin{figure}[h]
\centering
\includegraphics[scale=0.4]{ccii.png}
\caption[CCII symbol]{CCII symbol \cite{8}}
\end{figure}
\noindent Vstupní impedance na vstupu Y je nekonečná (tedy proud tekoucí skrz Y je nulový) a impedance na vstupu X je nulová ($R_Y = \infty, I_Y = 0, R_X = 0$). Napětí na vstupu X je ekvivalentní k napětí na vstupu Y ($V_X = V_Y$). Proud procházející vstupem X je ekvivalentní k proudu vstupem Z ($I_Z = I_X$). Výstupní impedance vstupu Z je nekonečná ($R_Z = \infty$).
Charakteristika ideálního \textit{CC} je reprezentována maticí
\begin{equation}
\begin{bmatrix}
I_Y \\ V_X \\ I_Z
\end{bmatrix}
=
\begin{bmatrix}
0 & 0 & 0 \\
1 & 0 & 0 \\
0 & \pm 1 & 0 
\end{bmatrix}
\begin{bmatrix}
V_Y \\
I_X \\
V_Z
\end{bmatrix}.
\end{equation}
\begin{figure}[h]
\centering
\includegraphics[scale=0.45]{cciiota.png}
\caption[CCII s $\pm$ výstupem založený na OTA]{CCII s $\pm$ výstupem založený na OTA \cite{8}}
\end{figure}
\noindent Takovéto zapojení funguje jako dobrý sledovač napětí, ale zato má menší šířku pásma.\\
\newline
Využitím zapojení na obrázku \ref{s:OTA} a principů \textit{CCII} lze získat modifikace klasického transkonduktančního zesilovače (s rozdílovým stupněm na vstupu a jedním výstupem). Obdržené atypické struktury obsahují jeden vstup a jeden výstup a také dva rozdílové stupně (na vstupu i výstupu).
\begin{figure}[h]
\centering
\includegraphics[scale=0.6]{siso.png}
\caption[Single input single output OTA (SISO) založený na CCII]{Single input single output OTA (SISO) založený na CCII \cite{8}}
\end{figure}
\begin{figure}[h]
\centering
\includegraphics[scale=0.6]{dido.png}
\caption[\textit{Fully differential} OTA (DIDO) založený na CCII a napěťovém bufferu]{\textit{Fully differential} OTA (DIDO) založený na CCII a napěťovém bufferu \cite{8}}
\end{figure}
\newpage
\subsection{Integrované obvody s OTA}
Integrované obvody se vyrábí buď s jedním nebo dvěma zesilovači v pouzdře. Varianty s jedním operačním zesilovačem jsou např. OPA615, OPA860 a novější OPA861. Všechny součástky s jedním OTA mají velkou šířku pásma (v řádech stovek MHz), cenově vychází na 75--280 Kč. Integrované obvody s dvěma zesilovači v pouzdře mají užší šířku pásma (2 MHz), menší rychlost přeběhu (50 V/$\mu$s), mnohem menší výstupní proud (650 $\mu$A) i offset vstupního napětí a operují při cca 4x nižších proudech. Cenové rozpětí je 25--65 Kč.
\renewcommand{\arraystretch}{1.5}
\begin{table}[h]
\scalebox{0.9}{%
  \begin{tabular}{ | c | >{\centering\arraybackslash}p{2cm}| >{\centering\arraybackslash}p{1.5cm} | >{\centering\arraybackslash}p{1.5cm} | >{\centering\arraybackslash}p{1.25cm} | >{\centering\arraybackslash}p{1.5cm} | >{\centering\arraybackslash}p{1.75cm} | >{\centering\arraybackslash}p{2cm} | >{\centering\arraybackslash}p{1.75cm} |}
    \hline
      & GBP - Gain Bandwidth Product & SR - Slew Rate & Output Current per Channel & $I_b$ - Input Bias Current & $V_{os}$ - Input Offset Voltage & Operating Supply Current & Forward Transconductance Min & Supply Voltage\\ \hline
    OPA615 & 710 MHz & 2.5 kV/$\mu$s & 5 mA & 3 $\mu$A & 40 mV & 13 mA & 65 mA/V & 8--12.4 V \\ \hline
    OPA860 & 470 MHz & 3.5 kV/$\mu$s & 15 mA & 5 $\mu$A & 12 mV & 11.2 mA & 80 mA/V & 5--13 V \\ \hline
    OPA861 & 400 MHz & 900 V/$\mu$s & 15 mA & 1 $\mu$A & 12 mV & 5.4 mA & 65 mA/V & 4--12.6 V  \\
    \hline
  \end{tabular}}
  \caption[Porovnání integrovaných obvodů s jedním OTA]{\label{tab:Porovnání integrovaných obvodů s jedním OTA}Porovnání integrovaných obvodů s jedním OTA \cite{9}}
  \end{table}
\begin{center}
\begin{table}[h]
\scalebox{0.9}{%
  \begin{tabular}{ | c | >{\centering\arraybackslash}p{2cm}| >{\centering\arraybackslash}p{1.5cm} | >{\centering\arraybackslash}p{1.5cm} | >{\centering\arraybackslash}p{1.25cm} | >{\centering\arraybackslash}p{1.5cm} | >{\centering\arraybackslash}p{1.75cm} | >{\centering\arraybackslash}p{2cm} | >{\centering\arraybackslash}p{1.75cm} |}
    \hline
      & GBP - Gain Bandwidth Product & SR  - Slew Rate & Output Current per Channel & $I_b$ - Input Bias Current & $V_{os}$ - Input Offset Voltage & Operating Supply Current & Forward Transconductance - Min & Supply Voltage\\ \hline
    LM13700 & 2 MHz & 50 V/$\mu$s & 650 $\mu$A & 5 $\mu$A & 4 mV & 1.3 mA & 6700 $\mu$S & 10--36 V \\ \hline
    NE5517 & 2 MHz & 50 V/$\mu$s & 650 $\mu$A & 5 $\mu$A & 5 mV & 2.6 mA & 5400 $\mu$S & 4--44 V \\ \hline
    AU5517 & 2 MHz & 50 V/$\mu$s & 650 $\mu$A & 5 $\mu$A & 5 mV & 2.6 mA & 5400 $\mu$S & 4--44 V  \\ \hline
    NJM13600 & 2 MHz & 50 V/$\mu$s & 650 $\mu$A & 5 $\mu$A & 5 mV & 2.6 mA & 6700 $\mu$S & 36 V  \\ \hline
    NJM13700 & 2 MHz & 50 V/$\mu$s & 650 $\mu$A & 5 $\mu$A & 4 mV & 2.6 mA & 6700 $\mu$S & 36 V  \\ \hline
  \end{tabular}}
  \caption[Porovnání integrovaných obvodů se dvěma OTA]{\label{tab:Porovnání integrovaných obvodů se dvěma OTA}Porovnání integrovaných obvodů se dvěma OTA \cite{9}}
  \end{table}
\end{center}
\noindent Pro realizaci přeladitelného filtru byl zvolen LM13700.
\begin{figure}[h]
\centering
\includegraphics[scale=0.55]{image6.png}
\caption[Konfigurace pinů na LM13700M]{Konfigurace pinů na LM13700M \cite{10}}
\end{figure}
\noindent Vnitřní zapojení LM13700 na obrázku \ref{s:OTA} obsahuje symetrický rozdílový stupeň (tranzistory Q4, Q5), který je napájen řízeným zdrojem proudu s tranzistorem Q2. Dvojice diod a tranzistorů tvoří proudová zrcadla (\textit{Current Mirror}) - referenční proud tekoucí v jedné větvi obvodu se \uv{zrcadlí} ~v jeho druhé větvi. Principiálně jsou to zdroje proudu řízené proudem. 
\begin{figure}[h]
\centering
\includegraphics[scale=0.75]{image5.png}
\caption[Vnitřní schéma OTA]{Vnitřní schéma OTA \cite{10}\label{sec:OTA}}
\end{figure}